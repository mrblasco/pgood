\documentclass[12pt, titlepage]{article}
\usepackage[utf8]{inputenc}
\usepackage{amssymb}
\usepackage{amsmath}
\usepackage{amsfonts}
\usepackage{indentfirst} % Indent paragraph
\usepackage[norule,bottom]{footmisc} % Footing options
%\usepackage[justification=centering,textfont={sc},labelfont={rm}]{caption}

% Page settings
\usepackage[left=1.5in, right=1.5in, top=1in, bottom=1in]{geometry}
\usepackage{setspace}
\onehalfspacing
%\doublespacing  % \singlespacing 

% Font
\usepackage{tgbonum}
	% times, palatino, lmodern, tgtermes
	% bookman, charter, tgschola, pslatex
%\renewcommand{\familydefault}{tgbonum}

% Tools
\usepackage{beamerarticle} % Not sure
\usepackage{todonotes} % Todos
\usepackage{appendix} % Appendix
\usepackage{array,booktabs,longtable,rotating} % Tables
%\usepackage{lineno}+ %\linenumbers

% Sections, captions, etc.
\usepackage{sectsty} % Section styles
\sectionfont{\centering\scshape} % \normalfont
\usepackage{titlesec}
\titlelabel{\thetitle.\quad}
%\subsectionfont{\itshape}
%\renewcommand{\thesection}{\arabic{section}.}
%\renewcommand{\thesubsection}{\thesection\arabic{subsection}.}

% Line numbers

% Links
\usepackage{hyperref}
%\hypersetup{%
%  colorlinks=false,% hyperlinks will be black
%  linkbordercolor=red,% hyperlink borders will be red
%  pdfborderstyle={/S/U/W 1}% border style will be underline of width 1pt
%}

% Position tables {here, top, bottom, page}
\makeatletter
\def\fps@table{htbp}
\makeatother

%% ... at the end of paper

% Create new minipage environment for notes 
% at the bottom of tables or figures
\newenvironment{tablenotes}[1][Note:]{
  \vskip 1.8ex
  \begin{minipage}{\textwidth}\itshape\footnotesize{#1}
} {\end{minipage}}


% Graphics
\usepackage{graphicx,grffile}
\makeatletter
\def\maxwidth{\ifdim\Gin@nat@width>\linewidth\linewidth\else\Gin@nat@width\fi}
\def\maxheight{\ifdim\Gin@nat@height>\textheight\textheight\else\Gin@nat@height\fi}
\makeatother
% Scale images if necessary, so that they will not overflow the page
% margins by default, and it is still possible to overwrite the defaults
% using explicit options in \includegraphics[width, height, ...]{}
\setkeys{Gin}{width=\maxwidth,height=\maxheight,keepaspectratio}
% set default figure placement to htbp
\makeatletter
\def\fps@figure{htbp}
\makeatother

\usepackage{natbib}% plainnat, abbrvnat
\bibliographystyle{plainnat}
\setcitestyle{authoryear,open={(},close={)}}
%\bibliographystyle{aer}


\setlength{\emergencystretch}{3em}  % prevent overfull lines
\providecommand{\tightlist}{%
  \setlength{\itemsep}{0pt}\setlength{\parskip}{0pt}}



\title{Incentives for Public Goods Inside Organizations: Field Experimental
Evidence\thanks{Blasco: Harvard Institute for Quantitative Social Science, Harvard
University, 1737 Cambridge Street, Cambridge, MA 02138 (email:
\href{mailto:ablasco@fas.harvard.edu}{\nolinkurl{ablasco@fas.harvard.edu}}).
Jung: Harvard Business School, Soldiers Field, Boston, MA 02163 (email:
\href{mailto:oliviajung@gmail.com}{\nolinkurl{oliviajung@gmail.com}}),
Lakhani: Harvard Business School, Soldiers Field, Boston, MA 02163, and
National Bureau of Economic Research (email:
\href{mailto:k@hbs.edu}{\nolinkurl{k@hbs.edu}}). Menietti: Harvard
Institute for Quantitative Social Science, Harvard University, 1737
Cambridge Street, Cambridge, MA 02138 (email:
\href{mailto:mmenietti@fas.harvard.edu}{\nolinkurl{mmenietti@fas.harvard.edu}}).
We gratefully acknowledge the financial support of the MacArthur
Foundation (Opening Governance Network), NASA Tournament Lab, and the
Harvard Business School Division of Faculty Research and Development.
This project would not have been possible without the support of Eric
Isselbacher, Julia Jackson, Maulik Majmudar and Perry Band from the
Massachusetts General Hospital's Healthcare Transformation Lab.}}
\author{Andrea Blasco \and Olivia S. Jung \and Karim R. Lakhani \and Michael Menietti}
\date{Last updated: 13 February, 2018}

\begin{document}
\maketitle
\begin{abstract}
We report results of a natural field experiment conducted at a medical
organization that sought contribution of public goods (i.e., projects
for organizational improvement) from its 1200 employees. Offering a
prize for winning submissions boosted participation without affecting
the quality of the submissions. The effect was consistent across gender
and job type. We posit that the allure of a prize, in combination with
mission-oriented preferences, drove participation. Using a simple model,
we estimate that these preferences explain about a third of the
magnitude of the effect. We also find that the opportunity of winning
financial resources to lead one's own project implementation had a
negative effect on participation. These results were sensitive to the
solicited person's gender.

\smallskip\noindent 
JEL Classification: D23; H41; M52.

\smallskip\noindent 
Keywords: innovation contest; free rider problem; social preferences; altruism; idea generation; organization of work.
\end{abstract}


\clearpage
\tableofcontents
\setcounter{tocdepth}{2}
\clearpage

\section{Introduction}\label{introduction}

Workers employed by firms are often expected to perform activities that
go beyond their formal duties for the benefit of the organization, like
volunteering time in committees or suggesting improvements for workplace
practices and business processes. From the perspective of the
management, these ``organizational tasks'' are important because a
significant component of worker productivity is linked to complementary
organizational investments such as centralized information systems and
workers are often in a better position than managers to know areas that
need improvements and to find solutions {[}xxxx{]}.But spurring a high
engagement of employees can be hard. The main problem is that
contractual incentives are generally xxxx given the xxxx nature of this
type of work. As a result, employees are often asked to work on
organizational tasks with little, or no, direct compensation and may be
tempted to free-ride, hoping that other members of the organization will
pay the cost of the required effort.

From the perspective of the worker, contributing xxxx on the workplace
may also be beneficial as they do xx, yy, and zz. However, Complementary
organizational investments such as shared information systems, workplace
practices, and business processes account for a significant component of
the productivity of the workers employed by firms. This means that
improving the efficacy of these investments may have generalized
benefits for the whole organization. Identification of these
organizational improvements, however, can be elusive for the management.
And workers who are often in a better position to identify the areas
most in need of improvement and find solutions, must divide their time
between XXX and YYY, perhaps lacking the motivation to do work on the
latter type of tasks. As a consequence, organizations are often in a
position where they need to engage their frontline workers in
organizational innovation tasks through some kind of incentive scheme.

In this paper, we study {[}what drives participation{]} how internal
contests -- organization-wide competitions for prizes -- that seek
employee contributions to internal public goods (e.g., xxxx) can address
these problems. Contests are increasingly popular tools. But we yet lack
a full understanding of the reasons why contests are useful to foster
contributions to public goods insider organizations. Following the
literature on public good provision in economics, we hypothesize three
main channels by which an internal contest can remedy the incentives.

First, when the competition awards personal rewards to the winners, an
internal contest may generate two opposing externalities among workers:
a negative externality from trying to win the personal reward over the
other employees; and a positive externality from contributing effort to
tasks that generate public good effects for the whole organization. As
shown by \citet{morgan2000financing}, non-altruistic agents contribute
beyond the value of the awarded prize when (i) the chances of winning
are proportional to one's effort and (ii) the value of the prize is
fixed (independent of effort). it can be shown that even small personal
rewards relative to the cost of contributing will mitigate the
free-riding incentive (\citet{morgan2000financing}). For this to happen,
however, two conditions must be met: (1) fixed reward and (2) quasi
linear preferences, which may, or may not, be hard to meet in the field.
And until now no study has xxxx.

A second mechanism through which internal contest may spur participation
is linked to xxxx the competition distributes internal implementation
resources to the winners. In this case, employees may be more willing to
participate as they have no budget constrain. In addition, when the
implementation of a public good can still be more or less favorable for
certain groups within the organization, the opportunity to influence
xxxx of internal resources to projects that are useful to them (for the
next years) can be a stronger incentive than just winning a small
personal reward. \ldots{} (2) the competition awards implementation
resources to the best projects, thus giving workers the opportunity of
influencing the allocation of collective resources towards preferred
interventions (e.g., obtaining implementation money for own projects);
and

Third, an internal contest is also an opportunity for the management to
mark specific organizational needs to the workers. This request may then
trigger voluntary contributions from workers out of their underlying
motivations towards helping the organization achieve its goals xxxx
\citep{besley2005competition, rotemberg2006altruism}. Intrinsic
incentives may be ``altruism'' directed at specific individuals or
groups within the organization (horizontal social preferences) or at the
organization as a whole (vertical social preferences). There exist
studies in psychology that have used surveys to measure xxxx. Overall
they say that: preferences correlate weakly. \ldots{} may be affected by
a number of factors both \emph{tangible} such as costs of required
effort and \emph{intangible} like social preferences on the workplace
(xxxx). For example, workers employed by organizations with a
``mission'' like those pursuing xxxx (hospitals, schools, and
non-profits) may have a relatively higher propensity to do xxxx than
other workers, as they might feel an ``intrinsic satisfaction'' from
helping the organization achieve its goals. But it is also possible that
workers may have an incentive to free ride on one another, despite the
xxxx. More generally, it can be difficult for organizations to motivate
workers effectively since little is known about what drives workers'
decisions to work on tasks that are like public goods inside
organizations.

We empirically assess the relative effectiveness of these three
alternative incentives in the field. Using a \emph{natural field
experiment} among the staff members of a medical organization, we
compare the responses of the entire staff member who were xxxx four
different \emph{solicitation treatments.} These are email contest
announcements seeking participation in an organization-wide internal
contest about project proposals for organizational improvement.

All the four solicitation emails were identical except for the first
paragraph which was assigned at random to isolate different motives: (1)
the opportunity of winning an individual reward (PRIZE); (2) the
opportunity of winning implementation money to lead one's own
improvement project (FUND); (3) a generic call to improve the workplace
(WPLACE) with no direct individual reward for the winners; and (4) a
generic call to improve the care of patients (PCARE) with no direct
individual reward for the winners.

These solicitation treatments are expected to affect participation rates
and, in turn, affect the average quality of employee contributions.
Quality effects can be drive, for example, by the allure of winning a
prize may attract to the competition workers with low-quality proposals,
who wouldn't have participated if the xxxx. At the same time, the
presence of compensation can distract efforts from intrinsically
motivated agents (as in xxxx). The analysis is then focused on two main
outcomes: (a) the decision to submit a proposal and engage in an
organizational improvement task and (b) the quality of the submissions
as measured by over 12,000 peer ratings and by the contest organizers
(``the management'').

Internal contests are increasingly popular sources of incentives inside
firms. In xxx, Levi's sought employees ideas on how to use electronic
spreadsheets to do xxxx. More recently, in 2016, Apple launched an
internal contest among its over xxx store employees seeking ideas on how
to improve the way it sells iPhones (xxxx); Xerox's internal contest
sought ideas from its employees on how to make the workplace a more
environmentally friendly; and IBM sought ideas of apps to xxxx.

Economists seem to agree that internal contests (also known as
tournaments) may have an incentive effect on workers' effort {[}xxx{]},
participation {[}xxx{]}, and help allocating resources to the best
project {[}xxxx{]}. However, much of the theoretical and empirical
literature on the topic presumes the absence of public good effects
among competitors; in the presence of which contests may no longer be
effective solutions \citep{drago1988incentive}. Also, contests may lead
to sorting problems xxxx, which are bad in the context of organizational
public goods.

In this study, we investigate empirically the relative effectiveness of
these incentives to better understand how organization-wide contests can
foster contributions to public goods inside organizations. We use a
\emph{natural field experiment} to compare the responses of staff
members of a medical organization who are exposed to four different
``solicitation'' treatments -- contest announcements -- seeking projects
for organizational improvement. These are: (1) a solicitation treatment
(PRIZE) announcing the opportunity of winning a prize for the best
proposals; (2) a solicitation treatment (WPLACE) announcing a generic
opportunity of improving the workplace; (3) a solicitation treatment
(PCARE) announcing a generic opportunity of improving the care of
patients; and (4) a solicitation treatment (FUND) announcing the
opportunity of obtaining implementation money to lead one's own
improvement project.

The empirical context for the experiment is the Massachusetts General
Hospital's (MGH) Corrigan Minehan Heart Center (simply referred to as
the ``Heart Center'') a prominent medical organization in the United
States and a teaching hospital of the Harvard Medical School. The health
care delivery context is particularly relevant as the need for
organizational improvement and innovation is vastly noted
\citep{cutler2012reducing}. In addition, health care professionals are
commonly seen as willing to step beyond the boundaries of their
contractual duties to offer better care \citep{delfgaauw2005dedicated},
which makes the comparison of different incentives towards a public good
especially relevant and interesting.

The subject pool was the entire population of the Heart Center
consisting of over 1,200 staff members including physicians, nurses, and
administrative staff. The intervention was associated with an internal
contest aimed to improve the operations of the organization, in the
spirit of ``open innovation'' discussed in
\citet{terwiesch2008innovation}, \citet{lakhani2013prize}, and
\citet{glaeser2016predictive}. The contest solicited employees to submit
project proposals describing an existing problem and providing a
solution to address the problem. After the submission phase, the contest
invited all employees to read and rate each proposal on a five-point
scale. The winning proposal would receive funding for implementation,
implying additional costs and responsibilities from making a winning
proposal (e.g., providing further guidance or a direct involvement in
implementation).

Our solicitation treatments were randomly assigned to each staff member,
thus allowing us to obtain causal estimates of the effect of different
solicitation strategies on two main outcomes: (a) the decision to submit
a proposal and engage in an organizational improvement task and (b) the
quality of the submissions as measured by over 12,000 peer ratings and
by the contest organizers (``the management'').

We further check the effects on employee participation of differences in
preferences or other characteristics associated with the employee's
profession, gender, and position inside the organization. The presence
of sorting effects based on the gender, for example, may impact on the
extent and type of public goods provided, complicating the analysis of
the incentives substantially.

As we shall discuss in Section \ref{results}, our solicitation
treatments produce significant effects on employee participation in the
contest, with small and insignificant effects on the quality of the
proposals. In particular, our findings suggest that: (1) the opportunity
of winning a prize dominates all other incentives; (2) the opportunity
of leading implementation of one's own submitted project proposal, a
non-production task, is the least effective incentive and seems to be
perceived more as a cost than a reward; (3) the increase in
participation rates associated with the announcement of a prize is
without lowering the quality of submissions; and using a simple linear
public-good model, we estimate that (4) responses to the prize incentive
may go beyond the extrinsic value of the prize, consistently with our
theory of prizes as means to internalize public goods.

In addition, by looking at the sorting by gender, profession, and
position inside the organization, we find that: (5) solicitation
treatments with mission-oriented incentives may result in responses that
appear sensitive to the gender of the solicited person (women's response
to solicitations for improving patient care is higher than men's); and
(6) gender differences in preferences, such as competitive inclinations
or risk aversion, may not exert great influence on responses of workers
to the competition-for-prizes incentive (women's and men's response to
solicitations for prizes are the same).

The implications of these results for the provision of public goods
inside organizations are discussed in Section
\ref{summary-and-conclusions}.

\section{Literature}\label{literature}

Economists have long recognized that prize-based competitions are an
important source of incentives inside organizations
\citep{lazear1981rank, green1983comparison, nalebuff1983prizes, mary1984economic}.
Much of the existing theoretical literature in labor economics, however,
presumes that agents are motivated to compete solely on the basis of the
utility derived by winning one of the prizes. Less attention has been
devoted to situations in which a competitor's performance generates
public good effects for the other competitors, and the whole
organization; a notable exception is XXX. By focusing on incentives to
workers to do organizational tasks (tasks with public good effects for
the organization), we improve the existing literature in this direction.

for which there exist consistent findings across many different
empirical settings, including sport competitions
\citep{ehrenberg1990tournaments}, production competitions in firms
\citep{knoeber1994testing, terwiesch2008innovation}, and more recently
online competitions
\citep{boudreau2011incentives, boudreau2016performance}.

To be sure, free riding incentives inside organizations have been widely
studied in labor economics, especially in the context of team production
\citep{erev1993constructive, hamilton2003team, boning2007opportunity, gibbs2014field}.
However, our study differs from much of the existing literature in that
it focuses on an individual competition where the team component is
missing. That is, the public good dilemma comes from externalities
towards anyone in the organization, not just a set of identified team
members. It follows that one can remove from consideration conventional
team dynamics such as peer pressure, monitoring, reciprocity among team
members, and other kinds of social interactions that have been shown to
affect behavior in the presence of free riding incentives.

Our study is also related the literature in public economics that
studies prize-based mechanisms to foster the provision of public goods.
\citet{morgan2000funding} appears to be the first to note that
fixed-prize lotteries -- a special case also known as ``Tullock''
contest -- are widely used tools among non-profit fundraising firms,
showing conditions under which these may increase the provision relative
to voluntary contributions. This insight has spurred much attention in
public economics with several studies testing this idea empirically
\citep[see][ for a survey]{vesterlund2012voluntary}. However, as noted
by Vesterlund, the existing evidence on the profitability of lotteries
for charities is only mixed. Our work extends the existing literature on
the topic by focusing on an organizational setting where monetary
contributions are replaced by effort and the ``greater good'' is helping
the organization achieve its goals. Within this context, we find
evidence that fixed-prize contests are a profitable tool to foster
public good effects inside firms.

Finally, our work provides support to the incentive effect of
mission-oriented preferences -- inner satisfaction from helping the
organization achieve its goals --
\citep{akerlof2005identity, besley2005competition, delfgaauw2005dedicated, delfgaauw2008incentives, prendergast2007motivation, rotemberg2006altruism}
and social preferences at work
\citep{bandiera2005social, bandiera2008social, bandiera2013team, dellavigna2016estimating}.
According to this perspective, workers are motivated agents. They do
their work because they care about their co-workers, employers, and
customers. Theoretical models suggest different ways in which managers
can exploit these intrinsic motivations to raise individual levels of
participation and productivity. Here, we use announcing an internal
contest for organizational improvements to make these motivations
salient. We find that emphasizing mission-oriented motivations has
countervailing effects: positive for women and negative for men. While
this finding is consistent with altruism being an important driver of
effort inside organization, it also suggests that people are sensitive
to the framing and in ways that may be difficult to predict ex-ante.

\section{Analytical framework and
predictions}\label{analytical-framework-and-predictions}

In this section, we conceptualize an internal solicitation for
innovation project proposals to improve the operations of the
organization as a voluntary contribution mechanism for a public good.
Successful proposals are viewed as non-excludable because innovation
leads to improvements for everyone in the workplace (including customers
by increasing the quality and efficiency of the services provided).
Submitting a proposal requires costly effort by employees, such as the
time to identify a problem, form a proposal, write up a concise
description, and the potential for further involvement during proposal
implementation.

Consider a linear model of the utility of a typical employee who
contributes \(x\) and benefits from total contributions of
\(Y=\sum x\):\footnote{functionalform}

\begin{equation} \label{eq:utility}
  u(R,~ Y) =  \gamma Y + \delta x + \frac{x}{Y} R - c x.
\end{equation}

The benefits of contributing derive from three sources. First, there is
an altruistic benefit from the improved workplace, \(\gamma Y\). The
altruistic benefits are the crux of public goods. Only the existence of
an improved workplace is desired and the source of contributions is
irrelevant. Thus, everyone would prefer to free ride on others' efforts.
Second, participants have some chance of winning the contest and can
expect to derive benefits from the prizes, \(\frac{x}{Y} R\), where, for
simplicity, all efforts have an equal chance of being selected as the
winner, as in \citet{morgan2000financing}. The personal reward \(R\) can
be thought of as a pecuniary prize, but it could also be an increase in
prestige or recognition or any combination of the above. Finally,
employees may have an egoistic motivation for contributing ``per se,''
regardless of winning and the effect on others, which is captured by
\(\delta x\). This includes the case in which workers may derive a
personal satisfaction from contributing personally to the organization,
often called warm glow preferences for giving \citep{andreoni1995warm}.
Since we cannot observe the distinction between altruistic and warm-glow
motives in our empirical setup, we are going to impose later that these
preferences are such that \(\delta=0\).

Contributors incur some cost from developing and submitting a proposal,
\(c x\). If there are \(n\) employees the public goods dilemma arises
when \(\gamma+\delta < c < n\gamma+\delta\). Then no individual would
contribute without a reward as costs exceed individual benefits, but
everyone would be better off if everyone contributes.

Suppose contributing a proposal is a discrete choice by employees. An
employee can either contribute a single proposal \(x=1\) and receive
utility of

\begin{equation}
    u_1 = \gamma \hat Y + \delta + \sum_{k=1}^{n}\Pr(Y=k)\frac{R}{k}  - c, 
\end{equation}

where \(\hat Y\) denotes the expected level of contributions and
\(\Pr(Y=k)\) is the probability of having \(k\) total contributions. Or
they can contribute nothing \(x=0\) and receive utility of

\begin{equation}
  u_0 = \gamma (\hat Y - 1).
\end{equation}

If there are \(n\) employees, then the unique symmetric mixed-strategy
equilibrium is for each employee to contribute a proposal with
probability \(p>0\). After using the binomial probability for
\(\Pr(Y=k)\), the payoff-equating condition to find a mixed-strategy
equilibrium is:

\begin{equation} \label{eq: mixed-strategy}
  \frac{1- (1-p)^{n}}{n p} = (c- \gamma - \delta) / R.
\end{equation}

This equation admits one single solution \(p^*\) which cannot be
expressed explicitly. Using a first order Taylor expansion around \(p\),
the equilibrium probability can be approximated as follows:

\begin{equation} \label{eq: probability}
  p^*  \approx \frac{2 (R- c+\gamma +\delta )}{(n-1) R}. 
\end{equation}

The analysis of the above model is used to derive the following
predictions.

\begin{enumerate}
\def\labelenumi{\arabic{enumi})}
\item
  The probability of contributing a proposal to improving the
  organization is zero when the prize for winning is sufficiently small
  relative to the individual cost of effort minus the preference for the
  public good (i.e., \(R< c-\gamma +\delta\)).
\item
  The probability of contributing a proposal to improve the organization
  increases with the value of the prize for winning.
\item
  The probability of contributing a proposal to improve the organization
  increases with the extent of individual preference for the public good
  (\(\gamma+\delta\)).
\end{enumerate}

Now suppose that the public good \(Y\) constitutes the sum of innovation
projects to improve the organization. Imagine that the quality of each
project is randomly drawn from a discrete distribution, the same for
every contributor (every employee who contributes is assumed to be
equally likely to come up with a useful idea). Each proposal can be of
high quality with probability \(\nu\) and of low quality with
probability \(1-\nu\). If a proposal is of low quality, then the value
for the organization is normalized to zero. The quality of proposals is
learned only after the agent paid the cost of effort. Now the
equilibrium public good \(Y\) is not deterministic but follows a
binomial distribution with average \(E[Y] = p^{**} \nu n\), where the
equilibrium probability \(p^{**}\) can be derived as before with the
only difference being that it is also an increasing function of the
probability \(\nu\). This leads to the following prediction.

\begin{enumerate}
\def\labelenumi{\arabic{enumi})}
\setcounter{enumi}{3}
\tightlist
\item
  If the public good depends on the quality of each contribution and
  every agent is equally likely to make a proposal of high quality, then
  the higher the probability of contributing, the higher is the average
  public good.
\end{enumerate}

This framework can be extended to the case of individuals with
heterogeneous costs. In the appendix, we explicitly consider the case of
two types of individuals with different marginal costs of effort that
form two groups of equal size. The symmetric mixed-strategy equilibrium
is then characterized by the vector of probabilities of contributing
with a proposal \((p_1^\star, p_2^\star)\). Here, the analysis of the
payoff-equating conditions for the mixed-strategy equilibrium shows that
the higher the marginal cost of effort minus preference for
contributing, the lower the equilibrium probability of individuals
(i.e., \(p_1^\star > p_2^\star\) when \(c_1 < c_2\), and vice versa).
This leads the final prediction.

\begin{enumerate}
\def\labelenumi{\arabic{enumi})}
\setcounter{enumi}{4}
\tightlist
\item
  If individuals have heterogeneous costs, then the probability of
  contributing a proposal to improve the organization is higher for
  agents with lower costs (positive sorting).
\end{enumerate}

\section{Experimental Design}\label{experimental-design}

\subsection{The context}\label{the-context}

The Heart Center is a leading academic medical center specializing in
clinical cardiac care and research in the United States. Founded more
than a hundred years ago, the Heart Center serves thousands of patients
every year, occupies more than 35,000 square feet of office space, and
employs more than 1,200 people (nurses, physicians, researchers,
technicians, and administrative staff) scattered across several
buildings on the Massachusetts General Hospital's main campus in
downtown Boston and a few other satellite locations.

The study was in cooperation with the Heart Center's launch of the
Healthcare Transformation Lab (HTL),\footnote{\url{http://www.healthcaretransformation.org}}
an initiative aimed at developing innovative health care process
improvements to enhance the health care safety and delivery of the
hospital. The launch of the HTL was accompanied by the announcement of
an internal ``innovation contest,'' called the Ether Dome
Challenge\footnote{The name is taken from a historical place on MGH's
  main campus where the first public surgery using anesthesia was
  demonstrated in 1846.} that sought to engage all staff members to
participate.

The communication around the innovation contest highlighted the
opportunity for staff to help in the selection process of the ideas and
a commitment by the Heart Center Management that the leading ideas would
be provided appropriate resources so that they could be implemented. The
announcement on the contest's website read:

\begin{quote}
``If you've noticed something about patient experience, employee
satisfaction, workplace efficiency, or anything that could be improved;
if you've had an inspiration about a new way to safeguard health; or if
you simply have a cost-saving idea, then now is the time to share your
idea.''
\end{quote}

\begin{figure}
\centering
\caption{Timeline of the innovation contest}
\label{timeline}
\includegraphics{figures/methods_timeline-1.pdf}
\end{figure}

The innovation contest was divided into three main phases (Figure
\ref{timeline}): submission, the peer evaluation, and implementation
phase.

The first was a four-week submission phase. All staff members were
encouraged to identify one or more organizational problems and submit
proposals addressing them. Employee participation was voluntary. All
project submissions were done online via the website of the contest.
There was no limit to the project proposals to submit (proposals could
cover any issue within the organization, as described above), but each
proposal was limited to approximately 300 words to lower the costs of
entry and encourage broader participation. To ensure that treatment
effects could be isolated, identified, and matched to participants, team
submissions were not permitted. Limiting submissions to individual
participation allowed us to match each submitter's characteristics to
the randomly assigned treatment. It also lowered incentives to
communicate or exchange information with other employees. Also, the
website was designed to not provide any information about the status of
the contest during the submission period. In this way, decisions could
not be easily influenced by the perceived popularity of the contest or
previous submissions.

It followed a two-week peer evaluation phase in which all staff members
were invited to rate the merit and potential of submitted proposals on a
five-point rating scale. All evaluations were done online on the website
of the contest. Each signed-up employee was shown a list of anonymized
proposals to read and rate. Proposals were presented at random in
batches of 10 each. Each proposal was described by a title, a main
description of the problem to solve, and the proposal. Voting was then
introduced by the following text: ``Rate this idea'' followed by the
rating scale: 1-low; 2; 3; 4; 5-high. Ratings were kept confidential and
the website did not provide any feedback or any other kind of additional
information that might have influenced individual judgment until the
voting phase was over. Evaluators were free to decide how many (and
which) proposals to rate. Since these were presented in a random order,
every proposal had on average the same exposure to people asked to rate
its quality. Evaluators were offered a limited edition T-shirt as a
compensation for the effort in voting.

In the final implementation phase, employees having submitted proposals
highly rated by peers and judged as particularly promising by the HTL
staff were invited to submit a full proposal detailing plans for
implementation. Following evaluation by MGH senior leadership, top
proposals were selected to receive support and funding for
implementation. This final phase took a few months to complete,
essentially the time necessary to select and implement the best
projects.

\subsection{The design}\label{the-design}

Within this context, we designed a \emph{natural field experiment}
(staff members are unaware of being part of an experiment). The basic
idea of the experiment was to randomize the content of the communication
announcing the innovation contest to all staff members. The start of the
submission phase was indeed announced to everyone in a series of
personalized emails. A direct message was sent to each contact in the
list of employees' emails from our subject pool.

The content of this communication with a placeholder for our
solicitation treatment is reported below (a copy of the exact email is
in the Appendix).

\begin{quote}
Dear Heart Center team member,

\textbf{Submit your ideas to {[}TREATMENT HERE{]}}

The Ether Dome Challenge is your chance to submit ideas on how to
improve the MGH Corrigan Minehan Heart Center, patient care and
satisfaction, workplace efficiency and cost. All Heart Center Staff are
eligible to submit ideas online. We encourage you to submit as many
ideas as you have: no ideas are too big or too small!

Submissions will be reviewed and judged in two rounds, first by the
Heart Center staff via crowd-voting, and then by an expert panel.
Winning ideas will be eligible for project implementation funding in the
Fall of 2014!
\end{quote}

The first paragraph of the above message was randomized into \emph{four}
different solicitation treatments (the exact words are in Table
\ref{experimental-design}), thus creating as many treatment groups of
equal size (Table \ref{experimental-design}). The first group was given
a solicitation treatment (PRIZE) announcing the innovation contest as an
opportunity to win individual prizes (iPad mini's) for top submissions.
The second group was given a solicitation treatment (FUND) announcing
the contest as an opportunity to win a \$20,000 budget for developing
their project proposals. The other groups received solicitation
treatments announcing the contest as an opportunity to improve the
health care of their patients (PCARE) or the workplace (WPLACE).

\begin{table}
\centering
\caption{Experimental design}
\label{experimental-design}
\begin{tabular}{@{}lp{5cm}>{\raggedright}rr}
  \\[-1.8ex]\hline \hline \\[-1.8ex]
 & \multicolumn{1}{c}{\emph{Solicitation treatment:}}
                        & \multicolumn{2}{c}{\emph{Employees:}}\\
                        \cmidrule(lr){2-2}\cmidrule(lr){3-4} &   & freq. & \% \\ 
  \hline \\[-1.86ex]
PRIZE & Submit your ideas to win an Apple iPad mini & 312 & 25 \\ 
  [1.8ex] FUND & Submit your ideas to win project funding up to \$20,000 
            to turn your ideas into actions & 308 & 25 \\ 
  [1.8ex] PCARE & Submit your ideas to improve patient care at the Heart Center & 310 & 25 \\ 
  [1.8ex] WPLACE & Submit your ideas to improve the workplace at the Heart Center & 307 & 25 \\ 
  [1.8ex] Total &  & 1237 & 100 \\ 
   \\[-1.8ex]\hline \hline \\[-1.8ex]
\end{tabular}
\end{table}

A sample size of more than 300 units for each treatment group ensured a
sufficiently high statistical power based upon standard power
calculations on the difference of proportions. In testing the difference
of proportions between any two treatments, the probability of type-I
errors was slightly below \(0.80\) for \emph{small} differences at 5
percent significance level but higher than \(0.80\) for \emph{medium}
and \emph{large} differences at the more stringent 1 percent
significance level.\footnote{The definition of small, medium and large
  differences is given by \citet{cohen1992power}; e.g., a difference of
  5 percentage points of the pair \((0.05, 0.10)\) is considered a small
  effect: see \citet{cohen1992power} p.~158.}

Also, note the lack of a traditional ``control'' treatment in this
study. Since the experiment was run in a workplace, we were constrained
to carry out treatments having equal chances of being successful. This
prevented us from having a `null' treatment with no personalized
incentives messaging as a control group. Indeed, the analysis focused on
multiple comparisons of several unordered discrete treatments (e.g.,
prizes vs funding vs framing).\footnote{Nevertheless, if we were to
  think of one treatment as the benchmark against which to compare the
  others, the FUND treatment would be our best candidate because giving
  information about the size of available funding is the default option
  for announcing grant programs and was part of the HTL's initial design
  before our cooperation in the experiment.}

These solicitation messages were sent three times: at the launch of the
submission phase, eight days from the launch and two days before the end
of the submission phase of the challenge.

The website of the innovation contest had supporting information about
the available prizes, funding, and timing of the initiative. The website
also required an institutional email address to login. Using this
feature, we designed the website graphics and layout to reinforce the
effect of the announcement: the headings, background images, a short
video, and the space just below a ``submit your ideas'' button were
designed to show the exact same first paragraph of the solicitation that
the employee received by email (i.e., text in Table
\ref{experimental-design}).

The MGH management and the HTL staff members were blind to group
assignment, which prevented potential bias in the communication of the
innovation contest that was not under our direct control. We also made
an effort to create a ``safe'' environment for employees submitting
proposals by making clear (in the application form) that the identity of
the proponents was going to be kept private unless the employee
self-identified, so that management could not identify workers without
their consent.

Finally, we relied only on official channels for communication to
strengthen the effect of the announcement and signal legitimacy of the
contest. Each employee received the same exact solicitation email three
times: at the launch, eight days from the launch and two days before the
end of the submission phase of the challenge. Starting from the second
week of the submission phase, information booths, flyers, and posters
were used to encourage everyone to take part in the event and respond to
the email solicitation. These flyers and posters were based on a
generic, undifferentiated version of the solicitation email without the
text of the treatments.

\section{Data}\label{data}

Our subject pool is the entire population working at the Heart Center as
of the end of 2014, a total of 1,237 individuals. For each individual,
we have administrative data on the gender, the type of profession, and
whether they had a fixed office location or not. Additional,
complementary data are available for a limited group of 378 employees
(31 percent). These extra data have self-reported information about
employees' demographics, such as age and years of tenure at the Heart
Center, that were obtained from an online survey that was run about two
months before the launch of the innovation contest.

We report summary statistics for the different variables by solicitation
treatment (Table \ref{summary-statistics}), showing that these are
statistically balanced across groups. These also show that the large
majority (72 percent) of employees in our sample are women. This is due
to the high fraction of workers being nurses (52 percent) and the
presence of a gender separation by profession with nurses being
predominantly women (92 percent).

Nursing workers constitute about half of the sample, and the rest is
split almost equally between physicians and administrative workers.
Though we do not have data on income, there exist large differences in
earnings across these professions. According to the United States Bureau
of Labor Statistics, the median annual wage of a physician was \$187,200
in 2015, which is about 60 percent higher than the that of a registered
nurse (\$67,490) and about 70 percent higher than that of a laboratory
technician (\$38,970). It follows that, if staff members are motivated
by the extrinsic value of the prize alone, one should expect large
differences in participation rates across profession.

Finally, it is also important to remark that only half of the employees
(\(53\) percent) have fixed office locations, as they may be on duty in
multiple wards. However, more senior staff tend to have a fixed
location. So, within each profession, this measure can be viewed as a
proxy for the employee's position or status inside the organization.

\begin{table}
\centering
\caption{Summary statistics by treatment}
\label{summary-statistics}
\begin{tabular}{@{}lccccccc}
  \\[-1.8ex]\hline \hline \\[-1.8ex]
 & \multicolumn{4}{c}{\emph{Assigned treatments:}} 
                        & \multicolumn{2}{c}{\emph{All:}}\\
                        \cmidrule(lr){2-5}\cmidrule(lr){6-7} & FUND & PCARE & WPLACE & PRIZE & \% & Obs. & P-value \\ 
  \hline \\[-1.86ex]
Other & 30 & 30 & 26 & 32 & 29 & 362 & 0.84 \\ 
  MD/Fellow & 19 & 18 & 18 & 18 & 18 & 226 &  \\ 
  Nursing & 51 & 52 & 56 & 51 & 52 & 649 &  \\ 
  [1.86ex] Female & 69 & 70 & 75 & 75 & 72 & 890 & 0.16 \\ 
  Male & 31 & 30 & 24 & 26 & 28 & 347 &  \\ 
  [1.86ex] No office & 50 & 46 & 47 & 45 & 47 & 577 & 0.56 \\ 
  Office & 50 & 54 & 52 & 56 & 53 & 660 &  \\ 
   [1.86ex] Age* &&&&&&\\18-25 & 6 & 8 & 8 & 6 & 6 & 24 & 1.00 \\ 
  26-35 & 29 & 29 & 31 & 26 & 29 & 107 &  \\ 
  36-45 & 18 & 19 & 24 & 16 & 22 & 81 &  \\ 
  $>$45 & 44 & 46 & 51 & 45 & 42 & 157 &  \\ 
   [1.86ex] Tenure* &&&&&&\\$<$ 10 & 40 & 31 & 36 & 37 & 36 & 132 & 0.89 \\ 
  10-20 & 26 & 29 & 38 & 28 & 30 & 111 &  \\ 
  20-30 & 12 & 19 & 15 & 10 & 14 & 50 &  \\ 
  30-40 & 10 & 16 & 15 & 12 & 13 & 48 &  \\ 
  $>$40 & 10 & 4 & 8 & 8 & 8 & 28 &  \\ 
   \\[-1.8ex]\hline \hline \\[-1.8ex]
\end{tabular}
\begin{minipage}{\textwidth}\itshape\footnotesize
Note: This table reports the percentage of employees in our sample cross tabulated by the assigned treatment across the gender, profession, whether the employee had a fixed office location, age, and years of tenure at the Heart Center. For each categorical variable, the last column reports the p-value from a Pearson's Chi-squared test with the assigned treatment and the variable. The asterisk $^{\ast}$ indicates self-reported information obtained from an online survey polling  employees about two months before the launch of the innovation contest.
\end{minipage}
\end{table}

\section{Summary and conclusions}\label{summary-and-conclusions}

We report results of a natural field experiment conducted at a medical
organization that held an innovation contest seeking contribution of
public goods (i.e., projects for organizational improvement) from its
more than 1200 employees. The experiment tested incentives for
contributing by manipulating the content of emails soliciting staff
participation. We presented different incentives to participate in the
contest, such as a prize (PRIZE) for winning submissions, improving
patient care (PCARE), improving the workplace (WPLACE), and funding for
implementation (FUND). Each staff was randomly assigned to receiving an
email containing one of the four incentives.

We find that the PRIZE solicitation treatment boosts participation by
about 40 percent relative to the WPLACE and PCARE solicitation
treatments. The FUND solicitation treatment is the least effective. It
generates not only about three times less submissions than the PRIZE
solicitation treatment, but also less submissions than the WPLACE and
PCARE solicitation treatments.

These participation differences, we find, are without changing the
quality of the submissions as judged by peers and the
management.\footnote{We find good agreement (high positive correlation)
  in the assessed quality of proposals between peer ratings and the
  evaluations conducted by the management; thus suggesting incentives
  being sufficiently aligned.} The higher employee participation in the
PRIZE solicitation treatment does not seem to be driven by low-quality
submissions. Similarly, the lower employee participation in the FUND
solicitation treatment does not seem to be driven by high-quality
submissions. In other words, treatments that attracted more (or less)
participation resulted in proposals of comparable quality and content.

Taken together, these findings suggest that (1) the
competition-for-prizes incentive dominates mission-oriented incentives
and (2) the opportunity to lead implementation of one's own submitted
project proposal is a poor incentive. In addition, these effects
combined with the small (extrinsic) value of the prize relative to the
median income of the participants, the long odds of winning the prize,
the lack of differences in participation among professions, and the
foreseeable additional costs of winning may suggest that (3) the effect
of a prize competition on participation goes beyond the actual value of
the prize itself, suggesting workers have, in fact, internalized some of
the benefits of participating in an organizational task.

We also fined that, although the WPLACE and PCARE solicitation
treatments are equally effective on average, responses appear sensitive
to the gender of the solicited person. Women's participation is greater
when emphasizing the patient care whereas men's participation is
significantly lower, controlling for the profession and position inside
the organization. This finding suggests that gender may be an important
factor influencing sensitivity of responses to solicitations concerning
the organizational mission.

At the same time, only an insignificant gender-based differences with
respect to participation in the PRIZE solicitation treatment was found:
women's participation was slightly higher but not significant than
men's, all else being equal. This evidence indicates that gender
differences in preferences, such as competitive inclinations or risk
aversion, may not exert great influence on responses of workers to
contests inside organizations.

We believe these results have three main implications for comparable
organizations and, more broadly, the internal provision of public goods.

The first implication is that announcing a competition for an individual
prize foster workers' participation in organizational tasks beyond the
value of the awarded prize itself. That is, prizes generate two opposing
externalities that help workers internalize the public good effects of
their organizational contributions. This result is important because it
highlights a relatively less understood function of contests that is to
mitigate the free riding incentives on organizational tasks.

A second implication is that offering the opportunity to lead collective
projects can exacerbate the free riding incentives. This result may
appear contrary to intuition. In theory, one may benefit more from
leading a project than winning an iPad. For instance, one may use the
opportunity to signal project management skills to the management aiming
for a career advancement; or steer some of the resources towards assets
or problems that are relatively more beneficial to his or her situation
compared to the rest of the organization. If so, why a negative result?
We believe that the private benefits from winning were negligible in our
setting. First, the opportunity for a career advancement is small
because medical staff gets promoted on the basis of other parameters
(e.g., the quality of care provided). Second, the peer evaluation and
vetting by the management ensure that the winning contributions yield as
distributed benefits as possible. These aspects may have eliminated the
possible private benefits from leading a project, resulting in poor
participation rates. This result is important because projects need to
be lead by someone and making more resources available does not seem to
increase volunteers.

A third implication is that participation in organizational tasks is
sometimes triggered by mission-based preferences. Although we find
evidence that these preferences can be an effective incentive, we also
find gender-based selection effects that are difficult to predict
ex-ante. Our experiment does not provide any insights to better
interpret these differences. But a large literature has investigated
gender-based difference in preferences \citep[see][ for a
review]{croson2009gender} or difference in self-stereotypes
\citep{coffman2014evidence} that could explain some of these effects.
Yet, more experimentation is needed to understand the different drivers
in the field inside organizations.

A few limitations of this study deserve consideration. The first is that
the validity of our causal interpretation of the results rests on a few
conventional assumptions \citep{rubin1974estimating}. These include the
``no interference between units'' assumption. In our study, it is
possible that communication among staff assigned to different treatment
arms could have influenced decisions to participate. The magnitude of
this interference would depend on intensity of staff communication and
the density of social interactions. Both of which should be small
because (i) an individual competition may provide only weak incentives
for information sharing and (ii) the staff members are scattered across
multiple buildings on the hospital campus. Even so, a potential
inference bias may alter the results towards a null effect as
differences in employee participation should converge towards zero when
communication spreads the content of the different email solicitations.
This goes against our results. Moreover, by looking at the temporal
dynamics of submissions, we find no indication of a convergence in the
participation rates. Hence, the assumption of no interference seems
appropriate.

Another potential limitation is that staff members may have left the
solicitation email that was sent to them unopened or unread, thus
non-complying with the assigned solicitation treatment. As this kind of
noncompliance is almost entirely unobserved,\footnote{The email was sent
  using the internal messaging system of the Heart Center, which, at the
  time, was not collecting individual analytics.} the analysis follows
an \emph{Intention-To-Treat} (ITT) approach, discarding entirely any
information about the solicitation treatment actually received. The main
drawback of an ITT analysis is that it does not answer questions about
causal effects of the content of the solicitation itself, only about
causal effects of the assignment to a solicitation treatment.

Finally, our results have implications that extend beyond the specific
organization under study. While the choice of focusing on health care
workers may limit the generalizability of our results in some respect,
it should be noted that in the US alone health care spending accounts
for 17 percent of the GDP (in 2015). And, more generally, our study
results are also directly applicable to a variety of other professions
exposed to a public good dilemma (e.g., teachers, public servants,
researchers). In all these settings, our study suggests that contests
soliciting employee contributions and awarding an individual prize to
the winning contribution appear an effective way to foster the internal
provision of public goods inside organizations.

\renewcommand\refname{References}
\bibliography{refs.bib}

\end{document}