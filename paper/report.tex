\documentclass[12pt, titlepage]{article}
\usepackage[utf8]{inputenc}
\usepackage{amssymb}
\usepackage{amsmath}
\usepackage{amsfonts}
\usepackage{indentfirst} % Indent paragraph
\usepackage[norule,bottom]{footmisc} % Footing options
%\usepackage[justification=centering,textfont={sc},labelfont={rm}]{caption}

% Page settings
\usepackage[left=1.5in, right=1.5in, top=1in, bottom=1in]{geometry}
\usepackage{setspace}
\onehalfspacing
%\doublespacing  % \singlespacing 

% Font
\usepackage{tgbonum}
	% times, palatino, lmodern, tgtermes
	% bookman, charter, tgschola, pslatex
%\renewcommand{\familydefault}{tgbonum}

% Tools
\usepackage{beamerarticle} % Not sure
\usepackage{todonotes} % Todos
\usepackage{appendix} % Appendix
\usepackage{array,booktabs,longtable,rotating} % Tables
%\usepackage{lineno}+ %\linenumbers

% Sections, captions, etc.
\usepackage{sectsty} % Section styles
\sectionfont{\centering\scshape} % \normalfont
\usepackage{titlesec}
\titlelabel{\thetitle.\quad}
%\subsectionfont{\itshape}
%\renewcommand{\thesection}{\arabic{section}.}
%\renewcommand{\thesubsection}{\thesection\arabic{subsection}.}

% Line numbers

% Links
\usepackage{hyperref}
%\hypersetup{%
%  colorlinks=false,% hyperlinks will be black
%  linkbordercolor=red,% hyperlink borders will be red
%  pdfborderstyle={/S/U/W 1}% border style will be underline of width 1pt
%}

% Position tables {here, top, bottom, page}
\makeatletter
\def\fps@table{htbp}
\makeatother

%% ... at the end of paper

% Create new minipage environment for notes 
% at the bottom of tables or figures
\newenvironment{tablenotes}[1][Note:]{
  \vskip 1.8ex
  \begin{minipage}{\textwidth}\itshape\footnotesize{#1}
} {\end{minipage}}


% Graphics
\usepackage{graphicx,grffile}
\makeatletter
\def\maxwidth{\ifdim\Gin@nat@width>\linewidth\linewidth\else\Gin@nat@width\fi}
\def\maxheight{\ifdim\Gin@nat@height>\textheight\textheight\else\Gin@nat@height\fi}
\makeatother
% Scale images if necessary, so that they will not overflow the page
% margins by default, and it is still possible to overwrite the defaults
% using explicit options in \includegraphics[width, height, ...]{}
\setkeys{Gin}{width=\maxwidth,height=\maxheight,keepaspectratio}
% set default figure placement to htbp
\makeatletter
\def\fps@figure{htbp}
\makeatother

\usepackage{natbib}% plainnat, abbrvnat
\bibliographystyle{plainnat}
\setcitestyle{authoryear,open={(},close={)}}
%\bibliographystyle{aer}


\setlength{\emergencystretch}{3em}  % prevent overfull lines
\providecommand{\tightlist}{%
  \setlength{\itemsep}{0pt}\setlength{\parskip}{0pt}}



\title{Incentives for Public Goods Inside Organizations: Field Experimental
Evidence\thanks{Blasco: Harvard Institute for Quantitative Social Science, Harvard
University, 1737 Cambridge Street, Cambridge, MA 02138 (email:
\href{mailto:ablasco@fas.harvard.edu}{\nolinkurl{ablasco@fas.harvard.edu}}).
Jung: Harvard Business School, Soldiers Field, Boston, MA 02163 (email:
\href{mailto:oliviajung@gmail.com}{\nolinkurl{oliviajung@gmail.com}}),
Lakhani: Harvard Business School, Soldiers Field, Boston, MA 02163, and
National Bureau of Economic Research (email:
\href{mailto:k@hbs.edu}{\nolinkurl{k@hbs.edu}}). Menietti: Harvard
Institute for Quantitative Social Science, Harvard University, 1737
Cambridge Street, Cambridge, MA 02138 (email:
\href{mailto:mmenietti@fas.harvard.edu}{\nolinkurl{mmenietti@fas.harvard.edu}}).
We gratefully acknowledge the financial support of the MacArthur
Foundation (Opening Governance Network), NASA Tournament Lab, and the
Harvard Business School Division of Faculty Research and Development.
This project would not have been possible without the support of Eric
Isselbacher, Julia Jackson, Maulik Majmudar and Perry Band from the
Massachusetts General Hospital's Healthcare Transformation Lab.}}
\author{Andrea Blasco \and Olivia S. Jung \and Karim R. Lakhani \and Michael Menietti}
\date{Last updated: 20 February, 2018}

\begin{document}
\maketitle
\begin{abstract}
We report results of a natural field experiment conducted at a medical
organization that sought contribution of public goods (i.e., projects
for organizational improvement) from its 1200 employees. Offering a
prize for winning submissions boosted participation without affecting
the quality of the submissions. The effect was consistent across gender
and job type. We posit that the allure of a prize, in combination with
mission-oriented preferences, drove participation. Using a simple model,
we estimate that these preferences explain about a third of the
magnitude of the effect. We also find that the opportunity of winning
financial resources to lead one's own project implementation had a
negative effect on participation. These results were sensitive to the
solicited person's gender.

\smallskip\noindent 
JEL Classification: D23; H41; M52.

\smallskip\noindent 
Keywords: innovation contest; free rider problem; social preferences; altruism; idea generation; organization of work.
\end{abstract}


\clearpage
\tableofcontents
\setcounter{tocdepth}{2}
\clearpage

\section{Introduction}\label{introduction}

Public goods problems are pervasive in organizations. Employees are
expected to be effective and efficient on their own assigned tasks but
also to make \emph{voluntary} contributions that go beyond the role
specifications, such as those towards improving the overall operations
and performance of the shared enterprise.

A central question for many organizations that rely upon these types of
public goods to increase productivity is determining what motivates
their own members to self-select and volunteer time and effort into
activities beyond the line of duty. Career concerns and other formal
incentives play a key xxx explaining why employees sort into these
discretionary tasks. Even so, sometimes workers seem to be motivated by
xxxx.

Professors, for example, are expected to teach and publish research
(with pay, promotion, and tenure tied to performance in those
activities) and serve on various internal committees that directly
benefit the university and department operations, with indirect benefits
to committee members (XXX). Similarly in companies, employees are
expected to work on production activities and also to collaborate in
teams, contribute to common resources, provide feedback on strategy and
direction of the firm, and drive innovation efforts.

A central question for many organizations that rely upon these types of
public goods to increase productivity is determining what motivates
their own members to self-select and volunteer time and effort into
activities beyond the line of duty. Career concerns and other formal
incentives play a large role. Yet, in many settings employees are not
formal incentives and may be tempted to behave opportunistically and
free ride, hoping that the other members of the organization will pay
the cost of doing these tasks. It is possible also that, despite these
free riding incentives, employees will contribute out of their
pro-social preferences towards specific individuals, or groups, within
the organization (xxxx) or the organization as a whole (xxxx). A third
is that the empowerment xxxx. But what is it? Is their willingness to do
something? Or altruistic preferences? Understanding these questions is
important also to enable organizations to design better incentive
schemes. (Pure or impure altruism?) (gift exchange?) (care about their
contribution to the firm?)

In this study, we examine empirically these two perspectives within the
context of an organization running an internal, organization-wide
competition that seek to engage its employee in contributions to
organizational public goods (e.g., improvements for the workplace).
Organizations often use internal contests to engage the workforce in
collective production. Among many popular examples are Apple's contest
xxxx; etc. Indeed, contests can solve the problem of incentives in many
ways. One of our goals is to unbundle some of these mechanism. In
particular, we focus on the effects on participation and self-selection
of three aspects of the contest: (1) the presence of personal ,
financial awards for the winners; (2) the presence of organizational
resources for the winners' to lead the implementation of their own
projects; and (3) the presence of call to action appealing to
contributing for the organization's own good.

Empirical research on prize-mechanisms for encouraging public goods have
been studied in the laboaroty; or in the context of charity donations
{[}xxxx{]}. Results xxxx. We extend these findings to inside the org.
Internal contests both explicit (e.g., sales contests) or implicit
(e.g., the mechanism promotions) have been studied in the literature
{[}xxx{]}. However, common assumption is that workers compete for the
prize and there are no spillover effects due to the public good nature
of their contributions.

\textasciitilde{}\textsubscript{\textsubscript{\textsubscript{\textasciitilde{}}}}

The effective functioning of organizations -- both in business and
non-profit -- crucially depends on the willingness' of their members to
volunteer time and effort in tasks that go beyond their regular duties
and produce generalized benefits. Typical examples are attending
decisional committees, suggesting improvements in workplace practices,
and experimenting with new ideas. However, the concomitant presence of
private costs for the worker and generalized benefits makes often hard
to engage employees in such operations.

One common remedy is to use \emph{internal contests} --
organization-wide competitions -- that grant resources and formal
recognition to employees who self-select into these tasks. Key elements
of these contests are a direct call to action, personal awards for the
winners, and internal resources awards enabling the winners to perform
the task. From a theoretical point of view, these elements may affect
behavior through many distinct mechanisms that involve different
extrinsic and intrinsic motivations. For example, a call to address
specific organizational needs may stimulate workers' pro-social
preferences towards specific individuals, or groups, within the
organization or the organization as a whole; whereas a call focused on
personal awards may stimulate the worker's self-interest. Understanding
how xxx respond to these different incentives is crucial to xxx and
design better incentives.

Existing empirical studies, however, have focused on contests where xxx
xxx and contest organizers xxxx. Much less attention has been given to
contests where output benefits contestants xxxx.

The theoretical literature on contests also has focused on contests with
no spillovers. But public goods prize-mechanisms to foster public goods
suggest ``small rewards.'' However, the recognition and xxx from being
running a project maybe a better incentive than a small reward. Also,
small rewards may crowd-out other xxx.

Based on these insights, here we report the results of a \emph{natural
field experiment} to unbundle the different mechanisms XXX. That is, we
try to assess the relative effectiveness of three alternative
incentives: (1) personal awards; (2) resource-grants awards; and (3)
communication of organizational needs.

Another important reason to focus on xxxx. There exists an extensive
literature in economics on the use of contests as a source of incentives
inside firms {[}xxxx{]}. Much of the existing theoretical literature,
however, presumes that agents are motivated to compete on the basis of
the utility derived by winning personal awards. Less attention has been
devoted to situations in which a competitor's performance generates
public good effects for the other competitors, and the whole
organization (a notable exception is XXX). By focusing on incentives to
workers to do organizational tasks (tasks with public good effects for
the organization), we improve the existing literature in this direction.

Make

In particular, the experiment involved more than 1200 staff members of a
medical organization (doctors, nurses, and administrative staff) in the
United States. St

we compare the responses of the entire staff member who were xxxx four
different \emph{solicitation treatments.} These are email contest
announcements seeking participation in an organization-wide internal
contest about project proposals for organizational improvement. All the
four solicitation emails were identical except for the first paragraph
which was assigned at random to isolate different motives: (1) the
opportunity of winning an individual reward (PRIZE); (2) the opportunity
of winning implementation money to lead one's own improvement project
(FUND); (3) a generic call to improve the workplace (WPLACE) with no
direct individual reward for the winners; and (4) a generic call to
improve the care of patients (PCARE) with no direct individual reward
for the winners.

In this paper we study xxxx. An internal contest may remedy the
incentives by providing a formal recognition to the workers who sorted
xxx and achieved the xxxx. The design of the awards and the overall
intent of the competition may largely affect participation of the
different members of the organization, revealing workers' actual reasons
for contributing to the organization. Based on this insight, we conduct
a natural field experiment to assess the relative importance of xxx,
xxx, and xxx.

The choice of sorting on these tasks yields an immediate cost for the
worker but also benefits for specific individuals, or groups, within the
organization or the organization as a whole. This means that
non-altruistic workers may be tempted to behave opportunistically and
free ride, hoping that others will sort on these tasks. But, at the same
time, workers may feel compelled to go the extra mile at work for
organizations pursuing \emph{social public goods} (hospitals, schools,
and non-profits) because of the intrinsic satisfaction from contributing
to the organization for its own good.

An internal contest may remedy the incentives by providing a formal
recognition to top contributors. The design of the awards and the
overall intent of the competition may largely affect participation of
the different members of the organization, revealing workers' actual
reasons for contributing to the organization. Based on this insight, we
conduct a natural field experiment to assess the relative importance of
xxx, xxx, and xxx.

A first mechanism is the incentive effect of winning a personal reward.
Theory predicts that, even in the context of public good provision,
prize-based competitions are better than voluntary mechanisms. In
particular, as shown by xxx, non-altruistic agents will contribute
beyond the value of the awarded prize when (i) the chances of winning
are proportional to one's effort and (ii) the value of the prize is
fixed (independent of effort). This is because the internal contest may
generate two opposing externalities: a negative externality from trying
to win the personal reward over the other employees; and a positive
externality from contributing effort to the public good. Organizations
can hence use small awards to get more than xxx participation rates and
higher returns per dollar.

However, prizes can also generate negative sorting effects. Low-quality
ideas, which is a waste. Crowding-out which is even worst.

A second potential mechanism is to use contests to award internal
resources to xxxx. In this case, employees may be more willing to
participate as they xxxx. There might be psychological benefits from
``being in charge''. But also private aspects of the public good could
appropriate xxxx. have no budget constrain. In addition, when the
implementation of a public good can still be more or less favorable for
certain groups within the organization, the opportunity to influence
xxxx of internal resources to projects that are useful to them (for the
next years) can be a stronger incentive than just winning a small
personal reward. \ldots{} (2) the competition awards implementation
resources to the best projects, thus giving workers the opportunity of
influencing the allocation of collective resources towards preferred
interventions (e.g., obtaining implementation money for own projects).

A third potential mechanism is to appeal to xxxx. an internal contest is
also an opportunity for the management to mark specific organizational
needs to the workers. This request may then trigger voluntary
contributions from workers out of their underlying motivations towards
helping the organization achieve its goals xxxx
\citep{besley2005competition, rotemberg2006altruism}. Intrinsic
incentives may be ``altruism'' directed at specific individuals or
groups within the organization (horizontal social preferences) or at the
organization as a whole (vertical social preferences). There exist
studies in psychology that have used surveys to measure xxxx. Overall
they say that: preferences correlate weakly. \ldots{} may be affected by
a number of factors both \emph{tangible} such as costs of required
effort and \emph{intangible} like social preferences on the workplace
(xxxx). For example, workers employed by organizations with a
``mission'' like those pursuing xxxx (hospitals, schools, and
non-profits) may have a relatively higher propensity to do xxxx than
other workers, as they might feel an ``intrinsic satisfaction'' from
helping the organization achieve its goals. But it is also possible that
workers may have an incentive to free ride on one another, despite the
xxxx. More generally, it can be difficult for organizations to motivate
workers effectively since little is known about what drives workers'
decisions to work on tasks that are like public goods inside
organizations.

Workers employed by firms are often expected to perform
\emph{organizational tasks} -- i.e., activities that go beyond their
regular duties and produce generalized benefits for the organization --
like volunteer time in committees, indicate improvements for workplace
practices, innovate.\footnote{EXAMPLE OF PROFESSORS} For the
management's perspective, it is central to engage the workforce in these
tasks because different workers are often in a better position to
identify areas that need improvements and find the solutions; although
it might be difficult to do so with formal incentives (xxxx). But also
from the employees' perspective, being engaged in organizational tasks
is advantageous to xxxx direction of the firm and contribute and xxx
promote change xxxx, although, of course, it can also be costly,
especially when it subtracts time to their formal duties. This
concomitance of generalized benefits and individual costs may create a
public good dilemma; workers may be tempted to free-ride, hoping that
other members of the organization will pay the cost of the required
effort; and may need to be encouraged through some kind of incentive
scheme.

In this paper, we focus on internal contests -- organization-wide
competitions for prizes {[}xxxx{]} -- as a way to address this problem
and raise the levels of participation and engagement of employees in
tasks that produce public good effects for the entire organization. In
other words, we investigate the ways in which an internal competition
that seek contributions to common goods can remedy the free-riding
incentives when workers face a public good dilemma.

The focus on internal contests is in part justified by the growing
popularity of contests as a source of incentives inside firms. Among the
many examples that have appeared in the news are the Apple's 2016
contest among its over xxx store employees seeking ideas on how to
improve the way it sells iPhones; Xerox's 20XX internal contest seeking
employees ideas on how to make a more environmentally friendly workplace
environment; IBM's 20xxx contests on employee ideas to improve xxxx
internal communication; and AT\&T's xxx contests seeking employee ideas
about new products {[}xxxx{]}.\footnote{See: ``Apple seeks `pie in the
  sky' ideas for innovation,''
  \href{http://www.computerworld.com/article/2474058/smartphones/apple-seeks--pie-in-the-sky--ideas-for-innovation}{Computerworld};
  ``Xerox employees green ideas save company \$10.2 million,''
  \href{http://www.theguardian.com/sustainable-business/xerox-employees-green-ideas-save}{The
  Guardian}; XXXXX, and ``AT\&T develops employee ideas for
  innovation,''
  \href{http://blogs.wsj.com/cio/2014/11/12/att-develops-employee-ideas-for-innovation}{The
  Wall Street Journal}).} And one may well expect this trend to
intensify as organizations do more xxxx and workers can xxxx.

Another motivation to focus on internal contests in a public good
context is to complement the existing economic literature on contests.
There exists indeed a large literature in economics on the use of
contests inside firms {[}\citet{lazear1981rank};
\citet{green1983comparison}; \citet{nalebuff1983prizes}; and
\citet{mary1984economic}; xxxxx{]}. Yet, the existing literature has
been mainly focused on tasks that have no public good effects. Here we
do xxxx. Despite its popularity, it is still unclear the mechanism
through internal contests. xxxx. Specifically, we hypothesize and test
empirically three main mechanisms by which an internal contest can
remedy the incentives in this case.

First, the incentive effect of competition is higher when there are
public good xxxx. As shown by \citet{morgan2000financing},
non-altruistic agents engaged in a competition that seeks contributions
to public goods and awards personal rewards to the winners (i.e., the
best contributors) will contribute beyond the value of the awarded prize
when (i) the chances of winning are proportional to one's effort and
(ii) the value of the prize is fixed (independent of effort). This is
because the internal contest may generate two opposing externalities: a
negative externality from trying to win the personal reward over the
other employees; and a positive externality from contributing effort to
the public good. Therefore, even small personal rewards relative to the
cost of contributing will mitigate the free-riding incentive.

This contrasts with the view of xxxx where the contest fail because of
the pg.

A second potential incentive is to award resources fxxxx. In this case,
employees may be more willing to participate as they xxxx. There might
be psychological benefits from ``being in charge''. But also private
aspects of the public good could appropriate xxxx. have no budget
constrain. In addition, when the implementation of a public good can
still be more or less favorable for certain groups within the
organization, the opportunity to influence xxxx of internal resources to
projects that are useful to them (for the next years) can be a stronger
incentive than just winning a small personal reward. \ldots{} (2) the
competition awards implementation resources to the best projects, thus
giving workers the opportunity of influencing the allocation of
collective resources towards preferred interventions (e.g., obtaining
implementation money for own projects).

A third potential mechanism is xxxxx. an internal contest is also an
opportunity for the management to mark specific organizational needs to
the workers. This request may then trigger voluntary contributions from
workers out of their underlying motivations towards helping the
organization achieve its goals xxxx
\citep{besley2005competition, rotemberg2006altruism}. Intrinsic
incentives may be ``altruism'' directed at specific individuals or
groups within the organization (horizontal social preferences) or at the
organization as a whole (vertical social preferences). There exist
studies in psychology that have used surveys to measure xxxx. Overall
they say that: preferences correlate weakly. \ldots{} may be affected by
a number of factors both \emph{tangible} such as costs of required
effort and \emph{intangible} like social preferences on the workplace
(xxxx). For example, workers employed by organizations with a
``mission'' like those pursuing xxxx (hospitals, schools, and
non-profits) may have a relatively higher propensity to do xxxx than
other workers, as they might feel an ``intrinsic satisfaction'' from
helping the organization achieve its goals. But it is also possible that
workers may have an incentive to free ride on one another, despite the
xxxx. More generally, it can be difficult for organizations to motivate
workers effectively since little is known about what drives workers'
decisions to work on tasks that are like public goods inside
organizations.

We empirically assess the relative effectiveness of these three
alternative incentives in the field. Using a \emph{natural field
experiment} among the staff members of a medical organization, we
compare the responses of the entire staff member who were xxxx four
different \emph{solicitation treatments.} These are email contest
announcements seeking participation in an organization-wide internal
contest about project proposals for organizational improvement. All the
four solicitation emails were identical except for the first paragraph
which was assigned at random to isolate different motives: (1) the
opportunity of winning an individual reward (PRIZE); (2) the opportunity
of winning implementation money to lead one's own improvement project
(FUND); (3) a generic call to improve the workplace (WPLACE) with no
direct individual reward for the winners; and (4) a generic call to
improve the care of patients (PCARE) with no direct individual reward
for the winners.

The empirical context for the experiment is the Massachusetts General
Hospital's (MGH) Corrigan Minehan Heart Center (simply referred to as
the ``Heart Center'') a prominent medical organization in the United
States and a teaching hospital of the Harvard Medical School. The health
care delivery context is particularly relevant as the need for
organizational improvement and innovation is vastly noted
\citep{cutler2012reducing}. In addition, health care professionals are
commonly seen as willing to step beyond the boundaries of their
contractual duties to offer better care \citep{delfgaauw2005dedicated},
which makes the comparison of different incentives towards a public good
especially relevant and interesting.

The subject pool was the entire population of the Heart Center
consisting of over 1,200 staff members including physicians, nurses, and
administrative staff. The intervention was associated with an internal
contest aimed to improve the operations of the organization, in the
spirit of ``open innovation'' discussed in
\citet{terwiesch2008innovation}, \citet{lakhani2013prize}, and
\citet{glaeser2016predictive}. The contest solicited employees to submit
project proposals describing an existing problem and providing a
solution to address the problem. After the submission phase, the contest
invited all employees to read and rate each proposal on a five-point
scale. The winning proposal would receive funding for implementation,
implying additional costs and responsibilities from making a winning
proposal (e.g., providing further guidance or a direct involvement in
implementation).

Within this empirical context, our solicitation treatments were randomly
assigned to each staff member, thus allowing us to obtain causal
estimates of the effect of different solicitation strategies on two main
outcomes: (a) employee participation measured by the decision to submit
a proposal and engage in an organizational improvement task and (b) the
quality of the submissions as measured by (over 12,000) peer ratings and
by the management organizing the contest.

Testing the presence of both participation and quality effects is
important as it would substantially complicate the problem of incentives
for the organization. Quality effects can arise when, for example, the
allure of winning a prize may attract to the competition workers with
low-quality proposals, who wouldn't have participated otherwise. At the
same time, the presence of compensation can discourage intrinsically
motivated agents who might xxxx; thus lowering quality of xxxx. More
generally, sorting is xxxx. So, we further check the effects on employee
participation of differences in preferences or other characteristics
associated with the employee's profession, gender, and position inside
the organization. The presence of sorting effects based on the gender,
for example, may impact on the extent and type of public goods provided,
complicating the analysis of the incentives substantially.

As we shall discuss in Section \ref{results}, results indicate that our
solicitation treatments produce significant effects on employee
participation in the contest, with small and insignificant effects on
the quality of the proposals. In particular, our findings suggest that:
(1) the opportunity of winning a prize dominates all other incentives;
(2) the opportunity of leading implementation of one's own submitted
project proposal, a non-production task, is the least effective
incentive and seems to be perceived more as a cost than a reward; (3)
the increase in participation rates associated with the announcement of
a prize is without lowering the quality of submissions; and using a
simple linear public-good model, we estimate that (4) responses to the
prize incentive may go beyond the extrinsic value of the prize,
consistently with our theory of prizes as means to internalize public
goods.

In addition, by looking at the sorting by gender, profession, and
position inside the organization, we find that: (5) solicitation
treatments with mission-oriented incentives may result in responses that
appear sensitive to the gender of the solicited person (women's response
to solicitations for improving patient care is higher than men's); and
(6) gender differences in preferences, such as competitive inclinations
or risk aversion, may not exert great influence on responses of workers
to the competition-for-prizes incentive (women's and men's response to
solicitations for prizes are the same).

The implications of these results for the provision of public goods
inside organizations are discussed in Section
\ref{summary-and-conclusions}.

\section{Literature}\label{literature}

Economists have long recognized that prize-based competitions are an
important source of incentives inside organizations
\citep{lazear1981rank, green1983comparison, nalebuff1983prizes, mary1984economic}.
Much of the existing theoretical literature in labor economics, however,
presumes that agents are motivated to compete solely on the basis of the
utility derived by winning one of the prizes. Less attention has been
devoted to situations in which a competitor's performance generates
public good effects for the other competitors, and the whole
organization; a notable exception is XXX. By focusing on incentives to
workers to do organizational tasks (tasks with public good effects for
the organization), we improve the existing literature in this direction.

for which there exist consistent findings across many different
empirical settings, including sport competitions
\citep{ehrenberg1990tournaments}, production competitions in firms
\citep{knoeber1994testing, terwiesch2008innovation}, and more recently
online competitions
\citep{boudreau2011incentives, boudreau2016performance}.

To be sure, free riding incentives inside organizations have been widely
studied in labor economics, especially in the context of team production
\citep{erev1993constructive, hamilton2003team, boning2007opportunity, gibbs2014field}.
However, our study differs from much of the existing literature in that
it focuses on an individual competition where the team component is
missing. That is, the public good dilemma comes from externalities
towards anyone in the organization, not just a set of identified team
members. It follows that one can remove from consideration conventional
team dynamics such as peer pressure, monitoring, reciprocity among team
members, and other kinds of social interactions that have been shown to
affect behavior in the presence of free riding incentives.

Our study is also related the literature in public economics that
studies prize-based mechanisms to foster the provision of public goods.
\citet{morgan2000funding} appears to be the first to note that
fixed-prize lotteries -- a special case also known as ``Tullock''
contest -- are widely used tools among non-profit fundraising firms,
showing conditions under which these may increase the provision relative
to voluntary contributions. This insight has spurred much attention in
public economics with several studies testing this idea empirically
\citep[see][ for a survey]{vesterlund2012voluntary}. However, as noted
by Vesterlund, the existing evidence on the profitability of lotteries
for charities is only mixed. Our work extends the existing literature on
the topic by focusing on an organizational setting where monetary
contributions are replaced by effort and the ``greater good'' is helping
the organization achieve its goals. Within this context, we find
evidence that fixed-prize contests are a profitable tool to foster
public good effects inside firms.

Finally, our work provides support to the incentive effect of
mission-oriented preferences -- inner satisfaction from helping the
organization achieve its goals --
\citep{akerlof2005identity, besley2005competition, delfgaauw2005dedicated, delfgaauw2008incentives, prendergast2007motivation, rotemberg2006altruism}
and social preferences at work
\citep{bandiera2005social, bandiera2008social, bandiera2013team, dellavigna2016estimating}.
According to this perspective, workers are motivated agents. They do
their work because they care about their co-workers, employers, and
customers. Theoretical models suggest different ways in which managers
can exploit these intrinsic motivations to raise individual levels of
participation and productivity. Here, we use announcing an internal
contest for organizational improvements to make these motivations
salient. We find that emphasizing mission-oriented motivations has
countervailing effects: positive for women and negative for men. While
this finding is consistent with altruism being an important driver of
effort inside organization, it also suggests that people are sensitive
to the framing and in ways that may be difficult to predict ex-ante.

\section{Analytical framework and
predictions}\label{analytical-framework-and-predictions}

In this section, we conceptualize an internal solicitation for
innovation project proposals to improve the operations of the
organization as a voluntary contribution mechanism for a public good.
Successful proposals are viewed as non-excludable because innovation
leads to improvements for everyone in the workplace (including customers
by increasing the quality and efficiency of the services provided).
Submitting a proposal requires costly effort by employees, such as the
time to identify a problem, form a proposal, write up a concise
description, and the potential for further involvement during proposal
implementation.

Consider a linear model of the utility of a typical employee who
contributes \(x\) and benefits from total contributions of
\(Y=\sum x\):\footnote{functionalform}

\begin{equation} \label{eq:utility}
  u(R,~ Y) =  \gamma Y + \delta x + \frac{x}{Y} R - c x.
\end{equation}

The benefits of contributing derive from three sources. First, there is
an altruistic benefit from the improved workplace, \(\gamma Y\). The
altruistic benefits are the crux of public goods. Only the existence of
an improved workplace is desired and the source of contributions is
irrelevant. Thus, everyone would prefer to free ride on others' efforts.
Second, participants have some chance of winning the contest and can
expect to derive benefits from the prizes, \(\frac{x}{Y} R\), where, for
simplicity, all efforts have an equal chance of being selected as the
winner, as in \citet{morgan2000financing}. The personal reward \(R\) can
be thought of as a pecuniary prize, but it could also be an increase in
prestige or recognition or any combination of the above. Finally,
employees may have an egoistic motivation for contributing ``per se,''
regardless of winning and the effect on others, which is captured by
\(\delta x\). This includes the case in which workers may derive a
personal satisfaction from contributing personally to the organization,
often called warm glow preferences for giving \citep{andreoni1995warm}.
Since we cannot observe the distinction between altruistic and warm-glow
motives in our empirical setup, we are going to impose later that these
preferences are such that \(\delta=0\).

Contributors incur some cost from developing and submitting a proposal,
\(c x\). If there are \(n\) employees the public goods dilemma arises
when \(\gamma+\delta < c < n\gamma+\delta\). Then no individual would
contribute without a reward as costs exceed individual benefits, but
everyone would be better off if everyone contributes.

Suppose contributing a proposal is a discrete choice by employees. An
employee can either contribute a single proposal \(x=1\) and receive
utility of

\begin{equation}
    u_1 = \gamma \hat Y + \delta + \sum_{k=1}^{n}\Pr(Y=k)\frac{R}{k}  - c, 
\end{equation}

where \(\hat Y\) denotes the expected level of contributions and
\(\Pr(Y=k)\) is the probability of having \(k\) total contributions. Or
they can contribute nothing \(x=0\) and receive utility of

\begin{equation}
  u_0 = \gamma (\hat Y - 1).
\end{equation}

If there are \(n\) employees, then the unique symmetric mixed-strategy
equilibrium is for each employee to contribute a proposal with
probability \(p>0\). After using the binomial probability for
\(\Pr(Y=k)\), the payoff-equating condition to find a mixed-strategy
equilibrium is:

\begin{equation} \label{eq: mixed-strategy}
  \frac{1- (1-p)^{n}}{n p} = (c- \gamma - \delta) / R.
\end{equation}

This equation admits one single solution \(p^*\) which cannot be
expressed explicitly. Using a first order Taylor expansion around \(p\),
the equilibrium probability can be approximated as follows:

\begin{equation} \label{eq: probability}
  p^*  \approx \frac{2 (R- c+\gamma +\delta )}{(n-1) R}. 
\end{equation}

The analysis of the above model is used to derive the following
predictions.

\begin{enumerate}
\def\labelenumi{\arabic{enumi})}
\item
  The probability of contributing a proposal to improving the
  organization is zero when the prize for winning is sufficiently small
  relative to the individual cost of effort minus the preference for the
  public good (i.e., \(R< c-\gamma +\delta\)).
\item
  The probability of contributing a proposal to improve the organization
  increases with the value of the prize for winning.
\item
  The probability of contributing a proposal to improve the organization
  increases with the extent of individual preference for the public good
  (\(\gamma+\delta\)).
\end{enumerate}

Now suppose that the public good \(Y\) constitutes the sum of innovation
projects to improve the organization. Imagine that the quality of each
project is randomly drawn from a discrete distribution, the same for
every contributor (every employee who contributes is assumed to be
equally likely to come up with a useful idea). Each proposal can be of
high quality with probability \(\nu\) and of low quality with
probability \(1-\nu\). If a proposal is of low quality, then the value
for the organization is normalized to zero. The quality of proposals is
learned only after the agent paid the cost of effort. Now the
equilibrium public good \(Y\) is not deterministic but follows a
binomial distribution with average \(E[Y] = p^{**} \nu n\), where the
equilibrium probability \(p^{**}\) can be derived as before with the
only difference being that it is also an increasing function of the
probability \(\nu\). This leads to the following prediction.

\begin{enumerate}
\def\labelenumi{\arabic{enumi})}
\setcounter{enumi}{3}
\tightlist
\item
  If the public good depends on the quality of each contribution and
  every agent is equally likely to make a proposal of high quality, then
  the higher the probability of contributing, the higher is the average
  public good.
\end{enumerate}

This framework can be extended to the case of individuals with
heterogeneous costs. In the appendix, we explicitly consider the case of
two types of individuals with different marginal costs of effort that
form two groups of equal size. The symmetric mixed-strategy equilibrium
is then characterized by the vector of probabilities of contributing
with a proposal \((p_1^\star, p_2^\star)\). Here, the analysis of the
payoff-equating conditions for the mixed-strategy equilibrium shows that
the higher the marginal cost of effort minus preference for
contributing, the lower the equilibrium probability of individuals
(i.e., \(p_1^\star > p_2^\star\) when \(c_1 < c_2\), and vice versa).
This leads the final prediction.

\begin{enumerate}
\def\labelenumi{\arabic{enumi})}
\setcounter{enumi}{4}
\tightlist
\item
  If individuals have heterogeneous costs, then the probability of
  contributing a proposal to improve the organization is higher for
  agents with lower costs (positive sorting).
\end{enumerate}

\section{Experimental Design}\label{experimental-design}

\subsection{The context}\label{the-context}

The Heart Center is a leading academic medical center specializing in
clinical cardiac care and research in the United States. Founded more
than a hundred years ago, the Heart Center serves thousands of patients
every year, occupies more than 35,000 square feet of office space, and
employs more than 1,200 people (nurses, physicians, researchers,
technicians, and administrative staff) scattered across several
buildings on the Massachusetts General Hospital's main campus in
downtown Boston and a few other satellite locations.

The study was in cooperation with the Heart Center's launch of the
Health-care Transformation Lab (HTL),\footnote{\url{http://www.healthcaretransformation.org}}
an initiative aimed at developing innovative health care process
improvements to enhance the health care safety and delivery of the
hospital. The launch of the HTL was accompanied by the announcement of
an internal ``innovation contest,'' called the Ether Dome
Challenge\footnote{The name is taken from a historical place on MGH's
  main campus where the first public surgery using anesthesia was
  demonstrated in 1846.} that sought to engage all staff members to
participate.

The communication around the innovation contest highlighted the
opportunity for staff to help in the selection process of the ideas and
a commitment by the Heart Center Management that the leading ideas would
be provided appropriate resources so that they could be implemented. The
announcement on the contest website read:

\begin{quote}
``If you've noticed something about patient experience, employee
satisfaction, workplace efficiency, or anything that could be improved;
if you've had an inspiration about a new way to safeguard health; or if
you simply have a cost-saving idea, then now is the time to share your
idea.''
\end{quote}

\begin{figure}
\centering
\caption{Timeline of the innovation contest}
\label{timeline}
\includegraphics{../figs/timeline.pdf}
\end{figure}

The innovation contest was divided into three main phases (Figure
\ref{timeline}): submission, the peer evaluation, and implementation
phase.

The first was a four-week submission phase. All staff members were
encouraged to identify one or more organizational problems and submit
proposals addressing them. Employee participation was voluntary. All
project submissions were done online via the website of the contest.
There was no limit to the project proposals to submit (proposals could
cover any issue within the organization, as described above), but each
proposal was limited to approximately 300 words to lower the costs of
entry and encourage broader participation. To ensure that treatment
effects could be isolated, identified, and matched to participants, team
submissions were not permitted. Limiting submissions to individual
participation allowed us to match each submitter's characteristics to
the randomly assigned treatment. It also lowered incentives to
communicate or exchange information with other employees. Also, the
website was designed to not provide any information about the status of
the contest during the submission period. In this way, decisions could
not be easily influenced by the perceived popularity of the contest or
previous submissions.

It followed a two-week peer evaluation phase in which all staff members
were invited to rate the merit and potential of submitted proposals on a
five-point rating scale. All evaluations were done online on the website
of the contest. Each signed-up employee was shown a list of anonymized
proposals to read and rate. Proposals were presented at random in
batches of 10 each. Each proposal was described by a title, a main
description of the problem to solve, and the proposal. Voting was then
introduced by the following text: ``Rate this idea'' followed by the
rating scale: 1-low; 2; 3; 4; 5-high. Ratings were kept confidential and
the website did not provide any feedback or any other kind of additional
information that might have influenced individual judgment until the
voting phase was over. Evaluators were free to decide how many (and
which) proposals to rate. Since these were presented in a random order,
every proposal had on average the same exposure to people asked to rate
its quality. Evaluators were offered a limited edition T-shirt as a
compensation for the effort in voting.

In the final implementation phase, employees having submitted proposals
highly rated by peers and judged as particularly promising by the HTL
staff were invited to submit a full proposal detailing plans for
implementation. Following evaluation by MGH senior leadership, top
proposals were selected to receive support and funding for
implementation. This final phase took a few months to complete,
essentially the time necessary to select and implement the best
projects.

\subsection{The design}\label{the-design}

Within this context, we designed a \emph{natural field experiment}
(staff members are unaware of being part of an experiment). The basic
idea of the experiment was to randomize the content of the communication
announcing the innovation contest to all staff members. The start of the
submission phase was indeed announced to everyone in a series of
personalized emails. A direct message was sent to each contact in the
list of employees' emails from our subject pool.

The content of this communication with a placeholder for our
solicitation treatment is reported below (a copy of the exact email is
in the Appendix).

\begin{quote}
Dear Heart Center team member,

\textbf{Submit your ideas to {[}TREATMENT HERE{]}}

The Ether Dome Challenge is your chance to submit ideas on how to
improve the MGH Corrigan Minehan Heart Center, patient care and
satisfaction, workplace efficiency and cost. All Heart Center Staff are
eligible to submit ideas online. We encourage you to submit as many
ideas as you have: no ideas are too big or too small!

Submissions will be reviewed and judged in two rounds, first by the
Heart Center staff via crowd-voting, and then by an expert panel.
Winning ideas will be eligible for project implementation funding in the
Fall of 2014!
\end{quote}

The first paragraph of the above message was randomized into \emph{four}
different solicitation treatments (the exact words are in Table
\ref{experimental-design}), thus creating as many treatment groups of
equal size (Table \ref{experimental-design}). The first group was given
a solicitation treatment (PRIZE) announcing the innovation contest as an
opportunity to win individual prizes (iPad mini's) for top submissions.
The second group was given a solicitation treatment (FUND) announcing
the contest as an opportunity to win a \$20,000 budget for developing
their project proposals. The other groups received solicitation
treatments announcing the contest as an opportunity to improve the
health care of their patients (PCARE) or the workplace (WPLACE).

\begin{table}
\centering
\caption{Experimental design}
\label{experimental-design}
\begin{tabular}{@{}lp{5cm}>{\raggedright}rr}
  \\[-1.8ex]\hline \hline \\[-1.8ex]
 & \multicolumn{1}{c}{\emph{Solicitation treatment:}}
						& \multicolumn{2}{c}{\emph{Employees:}}\\
						\cmidrule(lr){2-2}\cmidrule(lr){3-4} & 	 & freq. & \% \\ 
  \hline \\[-1.86ex]
PRIZE & Submit your ideas to win an Apple iPad mini & 312 & 25 \\ 
  [1.8ex] FUND & Submit your ideas to win project funding up to \$20,000 
			to turn your ideas into actions & 308 & 25 \\ 
  [1.8ex] PCARE & Submit your ideas to improve patient care at the Heart Center & 310 & 25 \\ 
  [1.8ex] WPLACE & Submit your ideas to improve the workplace at the Heart Center & 307 & 25 \\ 
  [1.8ex] Total &  & 1237 & 100 \\ 
   \\[-1.8ex]\hline \hline \\[-1.8ex]
\end{tabular}
\end{table}


A sample size of more than 300 units for each treatment group ensured a
sufficiently high statistical power based upon standard power
calculations on the difference of proportions. In testing the difference
of proportions between any two treatments, the probability of type-I
errors was slightly below \(0.80\) for \emph{small} differences at 5
percent significance level but higher than \(0.80\) for \emph{medium}
and \emph{large} differences at the more stringent 1 percent
significance level.\footnote{The definition of small, medium and large
  differences is given by \citet{cohen1992power}; e.g., a difference of
  5 percentage points of the pair \((0.05, 0.10)\) is considered a small
  effect: see \citet{cohen1992power} p.~158.}

Also, note the lack of a traditional ``control'' treatment in this
study. Since the experiment was run in a workplace, we were constrained
to carry out treatments having equal chances of being successful. This
prevented us from having a `null' treatment with no personalized
incentives messaging as a control group. Indeed, the analysis focused on
multiple comparisons of several unordered discrete treatments (e.g.,
prizes vs funding vs framing).\footnote{Nevertheless, if we were to
  think of one treatment as the benchmark against which to compare the
  others, the FUND treatment would be our best candidate because giving
  information about the size of available funding is the default option
  for announcing grant programs and was part of the HTL's initial design
  before our cooperation in the experiment.}

These solicitation messages were sent three times: at the launch of the
submission phase, eight days from the launch and two days before the end
of the submission phase of the challenge.

The website of the innovation contest had supporting information about
the available prizes, funding, and timing of the initiative. The website
also required an institutional email address to login. Using this
feature, we designed the website graphics and layout to reinforce the
effect of the announcement: the headings, background images, a short
video, and the space just below a ``submit your ideas'' button were
designed to show the exact same first paragraph of the solicitation that
the employee received by email (i.e., text in Table
\ref{experimental-design}).

The MGH management and the HTL staff members were blind to group
assignment, which prevented potential bias in the communication of the
innovation contest that was not under our direct control. We also made
an effort to create a ``safe'' environment for employees submitting
proposals by making clear (in the application form) that the identity of
the proponents was going to be kept private unless the employee
self-identified, so that management could not identify workers without
their consent.

Finally, we relied only on official channels for communication to
strengthen the effect of the announcement and signal legitimacy of the
contest. Each employee received the same exact solicitation email three
times: at the launch, eight days from the launch and two days before the
end of the submission phase of the challenge. Starting from the second
week of the submission phase, information booths, flyers, and posters
were used to encourage everyone to take part in the event and respond to
the email solicitation. These flyers and posters were based on a
generic, undifferentiated version of the solicitation email without the
text of the treatments.

\section{Data}\label{data}

Our subject pool is the entire population working at the Heart Center as
of the end of 2014, a total of 1,237 individuals. For each individual,
we have administrative data on the gender, the type of profession, and
whether they had a fixed office location or not. Additional,
complementary data are available for a limited group of 378 employees
(31 percent). These extra data have self-reported information about
employees' demographics, such as age and years of tenure at the Heart
Center, that were obtained from an online survey that was run about two
months before the launch of the innovation contest.

We report summary statistics for the different variables by solicitation
treatment (Table \ref{summary-statistics}), showing that these are
statistically balanced across groups. These also show that the large
majority (72 percent) of employees in our sample are women. This is due
to the high fraction of workers being nurses (52 percent) and the
presence of a gender separation by profession with nurses being
predominantly women (92 percent).

Nursing workers constitute about half of the sample, and the rest is
split almost equally between physicians and administrative workers.
Though we do not have data on income, there exist large differences in
earnings across these professions. According to the United States Bureau
of Labor Statistics, the median annual wage of a physician was \$187,200
in 2015, which is about 60 percent higher than the that of a registered
nurse (\$67,490) and about 70 percent higher than that of a laboratory
technician (\$38,970). It follows that, if staff members are motivated
by the extrinsic value of the prize alone, one should expect large
differences in participation rates across profession.

Finally, it is also important to remark that only half of the employees
(\(53\) percent) have fixed office locations, as they may be on duty in
multiple wards. However, more senior staff tend to have a fixed
location. So, within each profession, this measure can be viewed as a
proxy for the employee's position or status inside the organization.

\begin{table}
\centering
\caption{Summary statistics by treatment}
\label{summary-statistics}
\begin{tabular}{@{}lccccccc}
  \\[-1.8ex]\hline \hline \\[-1.8ex]
 & \multicolumn{4}{c}{\emph{Assigned treatments:}} 
						& \multicolumn{2}{c}{\emph{All:}}\\
						\cmidrule(lr){2-5}\cmidrule(lr){6-7} & FUND & PCARE & WPLACE & PRIZE & \% & Obs. & P-value \\ 
  \hline \\[-1.86ex]
Other & 30 & 30 & 26 & 32 & 29 & 362 & 0.84 \\ 
  MD/Fellow & 19 & 18 & 18 & 18 & 18 & 226 &  \\ 
  Nursing & 51 & 52 & 56 & 51 & 52 & 649 &  \\ 
  Female & 69 & 70 & 75 & 75 & 72 & 890 & 0.16 \\ 
  Male & 31 & 30 & 24 & 26 & 28 & 347 &  \\ 
  [1.86ex] No office & 50 & 46 & 47 & 45 & 47 & 577 & 0.56 \\ 
  Office & 50 & 54 & 52 & 56 & 53 & 660 &  \\ 
   [1.86ex] Age* &&&&&&\\
18-25 & 6 & 8 & 8 & 6 & 6 & 24 & 1.00 \\ 
  26-35 & 29 & 29 & 31 & 26 & 29 & 107 &  \\ 
  36-45 & 18 & 19 & 24 & 16 & 22 & 81 &  \\ 
  $>$45 & 44 & 46 & 51 & 45 & 42 & 157 &  \\ 
   [1.86ex] Tenure* &&&&&&\\
$<$ 10 & 40 & 31 & 36 & 37 & 36 & 132 & 0.89 \\ 
  10-20 & 26 & 29 & 38 & 28 & 30 & 111 &  \\ 
  20-30 & 12 & 19 & 15 & 10 & 14 & 50 &  \\ 
  30-40 & 10 & 16 & 15 & 12 & 13 & 48 &  \\ 
  $>$40 & 10 & 4 & 8 & 8 & 8 & 28 &  \\ 
   \\[-1.8ex]\hline \hline \\[-1.8ex]
\end{tabular}
\begin{minipage}{\textwidth}\itshape\footnotesize
Note: This table reports the percentage of employees in our sample cross tabulated by the assigned treatment across the gender, profession, whether the employee had a fixed office location, age, and years of tenure at the Heart Center. For each categorical variable, the last column reports the p-value from a Pearson's Chi-squared test with the assigned treatment and the variable. The asterisk $^{\ast}$ indicates self-reported information obtained from an online survey polling employees about two months before the launch of the innovation contest.
\end{minipage}
\end{table}


\section{Results}\label{results}

\subsection{Employee submissions}\label{employee-submissions}

We begin to evaluate the effect of the four experimental solicitation
treatments on employee participation in organizational public goods
provision by examining the percentage of employees who made project
submissions within the four-week submission period of the contest.

\begin{table}
\centering
\caption{Employee participation by solicitation treatment}
\label{submit}
\begin{tabular}{@{}lcccc}
  \\[-1.8ex]\hline \hline \\[-1.8ex]
 & \multicolumn{4}{c}{\emph{Solicitation treatment:}}\\
						\cmidrule(lr){2-5} Submission & FUND & PCARE & WPLACE & PRIZE \\ 
  \hline \\[-1.86ex]
No & 301 & 296 & 291 & 289 \\ 
  Yes &  7 & 14 & 16 & 23 \\ 
   \\[-1.8ex]\hline \hline \\[-1.8ex]
\end{tabular}
\end{table}


Table \ref{submit} shows the submission rates by solicitation treatment.
We test the independence between the participation rates and the four
solicitation treatments using a Fisher's exact test that gives a p-value
of 0.026. This means that the four experimental solicitation treatments
produced significant participation effects overall.

Pairwise comparisons among individual solicitation treatments show
further that: (1) the PRIZE solicitation treatment generates xxx, xxx,
and xxx times higher participation rates than the WPLACE, PCARE, and
FUND, respectively; (2) the PCARE and WPLACE solicitation treatments
generate basically identical participation rates; and (3) the FUND
solicitation treatment generates xx and xx times less participation
rates than xx and xxx.

\begin{table}
\centering
\caption{P-values for pairwise comparison of proportions}
\label{pairwise}
\begin{tabular}{@{}lccc}
  \\[-1.8ex]\hline \hline \\[-1.8ex]
 & FUND & PCARE & WPLACE \\ 
  \hline \\[-1.86ex]
PCARE & 0.124 &  &  \\ 
  WPLACE & 0.055 & 0.688 &  \\ 
  PRIZE & 0.003 & 0.132 & 0.269 \\ 
   \\[-1.8ex]\hline \hline \\[-1.8ex]
\end{tabular}
\begin{minipage}{\textwidth}\itshape\footnotesize
Note: This table reports the p-values of pairwise comparisons of proportions
						         among solicitation treatments.
\end{minipage}
\end{table}


To test to see whether these pairwise differences are statistically
significant, we use pairwise two-sample tests of proportions (Table
\ref{pairwise}). The analysis reveals that: the positive difference in
participation rates between the PRIZE and FUND solicitation treatments
is statistically significant (p=0.003); the positive difference between
the PRIZE and PCARE solicitation treatments is marginally significant
(p=0.132); whereas the positive difference between the PRIZE and WPLACE
solicitation treatments is insignificant (p=0.269); the negative
difference between the FUND and WPLACE solicitation treatments is
significant (p=0.055); and the negative difference between the FUND and
the PCARE solicitation treatments is marginally significant (p=0.124).
Overall, these findings are consistent with employee participation being
higher under a solicitation with personal awards incentives; and lower
under a solicitation with funding incentives.

\includegraphics{../figs/dynamics.pdf}

Next, we turn to examining participation dynamics (Figure
\ref{dynamics}). Though our data may not allow for a complete analysis
of participation dynamics, looking at the overall submission patterns
can be useful for the following reason. If employees assigned to
different solicitation treatments were sharing (either face-to-face or
electronically) the content of their solicitation with others, one
should expect participation rates to converge over time, yielding
estimates of the causal effects of a solicitation treatment biased
towards zero. Contrary to these expectations, we find no evidence of
convergence. Submissions in the PRIZE solicitation treatment are
constantly higher than in the other treatments (except perhaps in the
final week); at the same time, submissions in the FUND treatment are
constantly low. These patterns are hence consistent with communication
effects having little, or no, consequences on our findings, a topic we
will discuss in greater detail later (Section \ref{discussion}).

To complement the above analysis, we use a multiple linear regression
model that explicitly controls for observable differences across staff
members. The probability of submitting, \(y_i=1\), is given by:

\[\Pr(y_i=1) = \alpha_0 
                                    + \sum_{j} \alpha_{j} \text{SOLICIT}_{ij}
                                    + \text{JOB}_{i} 
                                    + \text{MALE}_{i} 
                                    + \text{OFFICE}_{i},\label{submit}\]

where \(\alpha_0\) is a constant, \(\alpha_j\) is the causal effect of
the solicitation treatment \(j\) assigned to an employee \(i\)
(\(\text{SOLICIT}_{ij}\)), controlling for the employee's profession
(\(\text{JOB}_i\)), the gender (\(\text{MALE}_i\)), and a dummy for
office location (\(\text{OFFICE}_i\)) indicating whether the employee
had a permanent office instead of being assigned to a ward. Notice that,
in our context, having a fixed office location is highly correlated with
the type of profession.{[}\^{}Much of the clinical staff might be mobile
and only half of the employees (\(53\) percent) had fixed office
locations, as they may be on duty in multiple wards. More senior staff
tend to have a fixed location. So, within each profession, this measure
can be viewed as a proxy for status inside the organization.{]} Nurses,
for instance, are more likely to being assigned to a ward than
physicians or administrative workers, due to the nature of their job.
Within each profession, however, having a fixed office location is
usually correlated with the hierarchical position inside the
organization. Beyond having a fixed office location per se, this
variable is hence potentially controlling for income and hierarchical
differences occurring within each profession as well.

\begin{table}
\centering
\caption{Probability of submitting proposals}\label{participation ols}
\begin{tabular}{@{\extracolsep{5pt}}lccccc} 
\\[-1.8ex]\hline 
\hline \\[-1.8ex] 
 & \multicolumn{5}{c}{\textit{Dependent variable:}} \\ 
\cline{2-6} 
\\[-1.8ex] & \multicolumn{5}{c}{ $SUBMIT_{ij}=1$ } \\ 
\\[-1.8ex] & (1) & (2) & (3) & (4) & (5)\\ 
\hline \\[-1.8ex] 
 PRIZE & 2.53$^{**}$ & 2.53$^{**}$ & 2.52$^{**}$ & 2.46$^{**}$ & 2.45$^{**}$ \\ 
  & (1.21) & (1.21) & (1.21) & (1.21) & (1.21) \\ 
  & & & & & \\ 
 WPLACE & 0.37 & 0.37 & 0.35 & 0.38 & 0.30 \\ 
  & (1.09) & (1.09) & (1.10) & (1.09) & (1.10) \\ 
  & & & & & \\ 
 FUND & $-$2.57$^{***}$ & $-$2.57$^{***}$ & $-$2.55$^{***}$ & $-$2.49$^{***}$ & $-$2.38$^{***}$ \\ 
  & (0.86) & (0.86) & (0.85) & (0.86) & (0.85) \\ 
  & & & & & \\ 
 Job (nursing) &  & 0.14 &  &  & 1.85 \\ 
  &  & (0.82) &  &  & (1.23) \\ 
  & & & & & \\ 
 Job (MD) &  & $-$0.31 &  &  & $-$1.14 \\ 
  &  & (1.03) &  &  & (1.24) \\ 
  & & & & & \\ 
 Male (yes) &  &  & $-$0.54 &  & $-$0.42 \\ 
  &  &  & (1.33) &  & (1.64) \\ 
  & & & & & \\ 
 Office (yes) &  &  &  & 2.79$^{**}$ & 4.56$^{***}$ \\ 
  &  &  &  & (1.20) & (1.60) \\ 
  & & & & & \\ 
 Constant & 4.84$^{***}$ & 4.78$^{***}$ & 5.00$^{***}$ & 3.35$^{***}$ & 1.97 \\ 
  & (0.61) & (0.66) & (0.73) & (0.75) & (1.25) \\ 
  & & & & & \\ 
\hline \\[-1.8ex] 
Log Likelihood & -5545 & -5545 & -5545 & -5542 & -5540 \\ 
Observations & 1,237 & 1,237 & 1,237 & 1,237 & 1,237 \\ 
\hline 
\hline \\[-1.8ex] 
\end{tabular} 
\begin{minipage}{\textwidth}
\emph{Note:} This table reports OLS estimates with heteroskedasticity robust standard errors in parenthesis. All coefficients are multiplied by 100 to indicate a percentage point change in the probability of submitting. Solicitation treatment dummies are coded to indicate deviations from the overall probability of submitting. The asterisks $^{\ast\ast\ast}$, $^{\ast\ast}$, $^{\ast}$ indicate significance at 1, 5 and 10 percent level, respectively.
\end{minipage}
\end{table}


The regression results (Table \ref{participation ols}) show an
insignificant effect on participation associated with an employee's
profession or gender.\footnote{The coefficient for nurses is positive
  and negative for physicians, consistent with sorting. These effects
  are, however, not statistically different from the residual category
  of other workers, as well as from one another.} By contrast, we find a
positive effect associated with the worker having a fixed office
location, as opposed to being assigned to a ward (and the effect size
doubles after controlling for the profession and gender). This evidence
suggests that differences in the employee's hierarchical position inside
the organization, as captured by our office-location regressor, may be a
stronger driver of participation relative to differences in gender and
profession as sometimes assumed by the literature.

The results of the regression model above (Table
\ref{participation ols}) give estimates of the solicitation treatment
differences relative to the overall mean participation and controlling
for baseline characteristics. In theory, these estimates should be more
statistically efficient relative to the pairwise comparisons above
(i.e., by reducing the overall noise associated with baseline
characteristics). We find that, at the 95 level of statistical
significance, employees in the PRIZE solicitation treatment are
\texttt{cf.full{[}"treatment2PRIZE"{]}} percentage points \emph{more}
likely to submit compared to the overall mean, whereas employees in the
FUND solicitation treatment are \texttt{-cf.full{[}"treatment2FUND"{]}}
percentage points \emph{less} likely to do so.\footnote{Subtracting
  these two effects gives
  \texttt{cf.full{[}"treatment2PRIZE"{]}\ -\ cf.full{[}"treatment2FUND"{]}}
  which is the difference in the probability of submitting between PRIZE
  and FUND treatments.} Overall, these effects are sizable not only in
absolute levels but also relative to the effects of the heterogeneity
captured by the other variables.

Following xxx literature (xxxx), we check whether gender was a factor
driving participation in the contest (with men relatively more willing
to sort into the competition). But we find no evidence (xxx gives a
p-value). Similarly we check whether the presence of effects associated
with unobservable characteristics that vary by job, like differences in
income and education. But we find again no evidence in favor of this
hypothesis (xxxx). We find instead evidence that having an office
location is xxx associated with participation. This result, however, is
confounded by other factors because more senior staff tend to have a
fixed location but, at the same time, having a fixed office location is
highly correlated with the type of profession (e.g., nurses xxxx).

We now turn to examining treatment interactions involving the employee's
gender and profession (Figure \ref{interactions}).\footnote{We find no
  significant differences for interactions with office location, which
  we do not report for space limitation.} We hypothesize gender
interactions to occur as a result of three main factors: differences in
risk taking, social preferences (willingness to contribute to public
goods), and competitive inclinations. If women prefer to work on
activities that are less risky, more pro-social (e.g., aiming at
improving people's health) and where competition is less intense, then
we should observe significant treatment interactions. Similarly, we
expect treatment interactions associated with the employee's profession
to occur because, for example, the prize opportunity (i.e., the PRIZE
treatment) could be relatively less effective for employees with a
higher income, such as doctors, than the others.

\begin{figure} 
\caption{Employee participation by gender or profession and solicitation treatment}
  \label{interactions}
  \centering
  \includegraphics{../figs/interactions.pdf}
\end{figure}

Examining the proportion of submissions conditional on the gender
(Figure \ref{interactions}, panel a) shows that women are more likely
(about 5 percentage points) to participate than men in the PCARE
solicitation treatment. And examining the same proportion conditional on
the profession (Figure \ref{interactions}, panel b) shows instead that
doctors are as likely to submit as any other worker in PRIZE
solicitation treatment; thus suggesting little sorting based on income
or other characteristics associated with a given profession.

\begin{table}
\centering
\caption{Gender differences}\label{tab: probability submitting interactions}
\begin{tabular}{@{\extracolsep{5pt}}lccc} 
\\[-1.8ex]\hline 
\hline \\[-1.8ex] 
 & \multicolumn{3}{c}{\textit{Dependent variable:}} \\ 
\cline{2-4} 
\\[-1.8ex] & \multicolumn{3}{c}{ $SUBMIT_{ij}=1$ } \\ 
\\[-1.8ex] & (1) & (2) & (3)\\ 
\hline \\[-1.8ex] 
 PRIZE$\times$female & 2.99$^{*}$ & 2.95$^{*}$ & 2.84 \\ 
  & (1.68) & (1.79) & (1.78) \\ 
  & & & \\ 
 PCARE$\times$female & 1.25 & 1.21 & 1.08 \\ 
  & (1.57) & (1.61) & (1.61) \\ 
  & & & \\ 
 FUND$\times$female & $-$2.91$^{***}$ & $-$2.95$^{**}$ & $-$2.79$^{**}$ \\ 
  & (1.06) & (1.20) & (1.19) \\ 
  & & & \\ 
 WPLACE$\times$female & $-$0.49 & $-$0.52 & $-$0.62 \\ 
  & (1.35) & (1.44) & (1.43) \\ 
  & & & \\ 
 PRIZE$\times$male & 1.37 & 1.42 & 1.40 \\ 
  & (2.44) & (2.51) & (2.50) \\ 
  & & & \\ 
 PCARE$\times$male & $-$3.75$^{***}$ & $-$3.72$^{***}$ & $-$3.64$^{***}$ \\ 
  & (1.15) & (1.16) & (1.16) \\ 
  & & & \\ 
 FUND$\times$male & $-$1.67 & $-$1.65 & $-$1.48 \\ 
  & (1.70) & (1.65) & (1.66) \\ 
  & & & \\ 
 Constant & 4.80$^{***}$ & 4.79$^{***}$ & 1.87$^{*}$ \\ 
  & (0.69) & (0.70) & (1.10) \\ 
  & & & \\ 
\hline \\[-1.8ex] 
Job & no & yes & yes \\ 
Office & no & no & yes \\ 
Log Likelihood & -5542 & -5542 & -5538 \\ 
Observations & 1,237 & 1,237 & 1,237 \\ 
\hline 
\hline \\[-1.8ex] 
\end{tabular} 
\begin{minipage}{\textwidth}
\emph{Note:} This table reports OLS estimates with heteroskedasticity robust standard errors in parenthesis. All coefficients are multiplied by 100 to indicate the percentage point change in the probability of submitting. Solicitation treatment dummies are coded to indicate deviations from the overall probability of submitting. The asterisks $^{\ast\ast\ast}$, $^{\ast\ast}$, $^{\ast}$ indicate significance at 1, 5 and 10 percent level, respectively.
\end{minipage}
\end{table}


To isolate gender and profession effects, we employ a version of model
\eqref{eq: submit} with gender-treatment interactions.\footnote{We also
  run a model with profession-treatment interactions and results are
  simular to those shown in Figure \ref{fig: interactions}.} The
regression results (Table
\ref{tab: probability submitting interactions}) show similar results to
the simple comparison of proportions. That is, after gradually adding
profession and office controls, interaction coefficients remain stable
across all specifications: the response of men under the PCARE
solicitation treatment is about 3 times the magnitude and in the
opposite direction of the women's response. By subtracting these two
coefficients, we find a significant difference between men and women of
about 5 percentage points (\(p=.018\)), which is consistent with our
previous analysis. Thus, and overall, men respond less than women in the
PCARE solicitation treatment, controlling for the profession and office
location. This effect could be due to gender differences in preferences,
as suggested by the literature, and we will return on this topic in the
discussion of the results.

\subsection{Employee participation in project
evaluation}\label{employee-participation-in-project-evaluation}

We now turn to examining the outcomes of the peer evaluation phase that
followed the submission phase of the contest. In this phase, 113 project
proposals ended up being rated by a total of \texttt{sum(raters)}
employees (\texttt{round(100*mean(raters),0)} percent of our sample) who
volunteered for the task. Their effort yielded a total of
\texttt{format(sum(ratings),\ big.mark=",")} evaluator-proposal pairs,
providing a very sensitive test for differences in project quality
across our solicitation treatments.

TABLE HERE

We check with linear regression whether the self-selected sample of
staff rating proposals is representative of the whole organization
(Table \ref{drivers_rating}), or just a subset of staff members. Testing
for statistical significance of the coefficients for the profession,
gender, ond office location shows that the evaluators are broadly
representative of the organization as a whole, albeit with a
significantly higher participation from staff members with an office
location. Furthermore, the lack of statistical significance for the
coefficients of the solicitation treatments shows that participation in
the evaluation phase was somewhat independent from the solicitation
treatment the staff members received in the submission phase.\footnote{One
  may find counterintuitive that there was less (although not
  significant) participation in the evaluation phase from employees in
  the PRIZE than in the other solicitation treatments, given the greater
  participation in the submission phase. This result is, however, not
  unexpected because only
  \texttt{round(100*mean(raters{[}submitters{]}))} percent of employees
  who made submissions resolved to rate proposals as well (we detect no
  difference in the propensity of submitting and rating proposals
  between the treatments); so, even a difference of 2 percentage points
  in submitting will shrink to about 1 percentage point in the rating
  phase. In other words, we expect self-rating to do not affect
  evaluation much.} Thus, and overall, the collected ratings appear a
profession-wide and gender-wide representative sampling of opinions
inside the organization.

\subsection{The quality of the project
proposals}\label{the-quality-of-the-project-proposals}

The treatment interventions may not have only impacted the propensity to
make a submission, but the quality of the submission as well. Of
particular interest is any indication of a quantity versus quality
trade-off. For example, if the treatment which generated the fewest
submissions (FUND) also produced the highest quality submissions. A
quality versus quantity trade-off would increase the complexity of
choosing optimal incentives for employees.

FIGURE RATINGS

TEST RATINGS

\emph{Quality assessed by peers.} To check whether differences in the
quality of the submissions can be explained by the solicitation
treatments of the submitter, we first look at differences in the
distribution of ratings obtained from peers. Overall, a project proposal
is given the ``neutral'' point (i.e., a rating of 3) on a five-point
scale about 30 percent of the times with employees being more likely to
give high (4-5) rather than low (1-2) ratings. This rating pattern does
not change much when we condition the data to the solicitation
treatments of the proponent (Figure \ref{ratings}); suggesting an equal
distribution of good and bad quality projects across the solicitation
treatments.

To formally test this hypothesis, we aggregate the mean ratings for each
proposal and regress these aggregate measures on solicitation treatment
dummies. The regression results (not reported) show only an
insignificant relationship between ratings and solicitation treatments.
The treatment coefficients are all insignificantly different from zero,
with the linear model not significantly different from a constant model
(an overall F-test gives a p-value of \texttt{ftest\$p.value}).

\begin{figure}
  \centering
  \caption{Distribution of ratings by solicitation treatment of the proponent}
  \label{ratings}
  \includegraphics{../figs/ratings.pdf}
\end{figure}

The above analysis on the aggregate ratings does not hold in
general.\footnote{It crucially relies on the assumption that an
  increment in a proposal's quality as measured by an increase in
  ratings from \(v\) to \(v+1\) is the same for any value \(v\).} So, we
also examine the distribution of ratings as generated by treatments with
no aggregation. We have over 12,000 ratings, providing a very sensitive
test for differences across treatments. Using a \texttt{cs\$method} we
find that the hypothesis of dependence between the distribution of
ratings and the treatments is \emph{not} quite significant at the 10
percent level (p-value of \texttt{cs\$p.value}). Driving the p-value is
a less than \(2\) percent difference between the proportion of 5's in
the WPLACE treatment versus the other distributions, which is probably
due to outliers (the winning proposal was in the WPLACE treatment).
Taken together with the fact that our sample is large, we have strong
evidence suggesting that there are no (economically meaningful)
differences in the quality of project proposals across treatments and in
particular no evidence of a quantity versus quality trade-off up to the
resolution of the five-point scale.\footnote{One may worry that such
  binning is a fairly coarse measure of quality. In particular, effects
  concentrated in the upper tail of the distribution may not be
  detected. For example, comparing the ratings of proposals A, B, C and
  D with hypothetical true qualities of 3, 4, 5, and 10 stars
  respectively. Under a five-point scale rating system, proposals A and
  B can be distinguished, but C and D cannot be distinguished. Hence,
  one needs to be very cautious in interpreting these results as
  evidence against quality effects in general.}

\begin{verbatim}
FINALISTS 
\end{verbatim}

\emph{Quality assessed by managers.} One potential limit of assessing
quality only on the basis of peer ratings is that the employees might
have a different view of a proposal's quality than executives (due, for
instance, to a misalignment of incentives). Indeed, to ensure alignment
between managerial goals and the peer assessment, all project proposals
were further vetted by the HTL staff before being considered for
implementation funding. So, we now focus on the outcomes of this vetting
process to investigate more broadly the presence of treatment effects on
the quality of project proposals.

The vetting process conducted by the HTL staff resulted in
\texttt{sum(z\$score\textgreater{}0)} proposals being scored (from 1 to
100 points) with the best \texttt{sum(hc\$finalist)} proposals invited
to submit implementation plans. The remaining \texttt{sum(z\$score==0)}
proposals were excluded (and received a score of zero) either because
flagged as inappropriate for funding or because the proponent manifested
no intention to participate in the implementation phase (a
\texttt{ft\$method} finds no association between proposals excluded and
treatments with a p-value of \texttt{ft\$p.value}).

The Spearman's rank correlation coefficient between the scores given by
the HTL staff and the average peer ratings was relatively high
(\texttt{spear}), indicating good agreement between our two measures of
quality. Indeed, as before, we find no treatment effects on quality
using the scores (a \texttt{kt\$method} gives a p-value of
\texttt{kt\$p.value}). We also find no treatment differences in the
percentage of submitters being selected and invited by HTL staff to
present additional implementation plans (a \texttt{ft\$method} gives a
p-value of \texttt{ft\$p.value}). Although not significant, employees
who made project proposals in the FUND solicitation treatment are less
likely to be selected as finalist than the others (only 1 out of 7 in
the FUND treatment were selected and invited by the HTL staff),
providing additional evidence of a no quantity versus quality trade-off,
as discussed before.

\subsection{The content of the project
proposals}\label{the-content-of-the-project-proposals}

The goal of the challenge was to improve Heart Center operations by
identifying problem areas and potential solutions. The proposed projects
broadly conformed to the stated goals of the contest, aligning with
improving the work processes within the organization or providing
high-quality patient care. For example, one project proposal that
received high peer ratings was to create a platform for patients to
electronically review and update their medicine list in the office prior
to seeing the physician. Another was to develop a smartphone application
showing a patient's itinerary for the day providing a guide from one
test or appointment to another. Nevertheless, other contest organizers
may have varying goals and be concerned about different aspects of the
submissions.

In order to examine additional dimensions of submission content, we now
study the area of focus of the submissions. Of particular interest is
understanding whether different wordings used in the general
encouragement solicitations (either towards improving the workplace or
targeting the wellbeing of patients) induce employees to concentrate on
different categories.

Members of the HTL categorized each project proposal into one of seven
``areas of focus'' (Table \ref{tab: area-of-focus}): three categories
(``Care coordination'', ``Staff workflow'', ``Workplace'') identified
improvements for the workplace, other three (``Information and access'',
``Patient care'', and ``Quality and Safety'') focused on improvements
centered around patients, and another one (``Surgical tools and support
to research'') categorized projects developing tools to support
scientific research.

We test overall association between these categories and the
solicitation treatments with a \texttt{ft\$method}. Results show a
marginally significant (p=\texttt{ft\$p.value}) association, which means
that our solicitation treatments have indeed an effect on the content of
the submitted proposals.

To test which areas of focus was affected by our treatment, we regress
the probability of a project proposal being in a given category against
solicitation treatment dummies. We use an F-test where the null
hypothesis tested is that all the treatment effects have a zero effect
on the probability of the proposal being in a given category. The
results of these F-tests of overall significance (Table
\ref{areas of focus}) reveals significant differences in the ``Quality
and Safety'' and ``Information and access'' categories, which we view as
improvements centered around patients (as opposed to workplace
improvements). The first significance result is due to project proposals
in the PCARE solicitation treatment being less likely to fall in the
``Quality and Safety'' category. The second result is due to project
proposals in the FUND solicitation treatment being less likely to fall
in the ``Information and access'' category.

Although it is difficult to interpret these results because our model
does not provide any prediction on the content of proposals, they
indicate a possible trade-off between stimulating participation via
solicitations and inducing selection in the type of contributions to the
public good, which complicates the analysis of incentives for public
goods inside organizations beyond what the current literature
anticipates.

We also look at differences in the underlying complexity of the project
proposal as captured by differences in the length (i.e., the word count)
of a submission. Submissions were below 200 words in most cases with
little differences between the treatments. Indeed, testing for a
significant linear regression relationship between the length of
submissions and treatment dummies returned an overall insignificant
result (p=.43, F-test).

As a result, based on the analysis of the areas of focus and the length
of the submissions, we do find only little evidence of differences in
submission content across treatments. However, submission content is not
a well-defined concept and could be characterized in many dimensions.
While content does not vary in the dimensions we selected, we have not
exhausted all possible dimensions.

\section{Summary and conclusions}\label{summary-and-conclusions}

We report results of a natural field experiment conducted at a medical
organization that held an innovation contest seeking contribution of
public goods (i.e., projects for organizational improvement) from its
more than 1200 employees. The experiment tested incentives for
contributing by manipulating the content of emails soliciting staff
participation. We presented different incentives to participate in the
contest, such as a prize (PRIZE) for winning submissions, improving
patient care (PCARE), improving the workplace (WPLACE), and funding for
implementation (FUND). Each staff was randomly assigned to receiving an
email containing one of the four incentives.

We find that the PRIZE solicitation treatment boosts participation by
about 40 percent relative to the WPLACE and PCARE solicitation
treatments. The FUND solicitation treatment is the least effective. It
generates not only about three times less submissions than the PRIZE
solicitation treatment, but also less submissions than the WPLACE and
PCARE solicitation treatments.

These participation differences, we find, are without changing the
quality of the submissions as judged by peers and the
management.\footnote{We find good agreement (high positive correlation)
  in the assessed quality of proposals between peer ratings and the
  evaluations conducted by the management; thus suggesting incentives
  being sufficiently aligned.} The higher employee participation in the
PRIZE solicitation treatment does not seem to be driven by low-quality
submissions. Similarly, the lower employee participation in the FUND
solicitation treatment does not seem to be driven by high-quality
submissions. In other words, treatments that attracted more (or less)
participation resulted in proposals of comparable quality and content.

Taken together, these findings suggest that (1) the
competition-for-prizes incentive dominates mission-oriented incentives
and (2) the opportunity to lead implementation of one's own submitted
project proposal is a poor incentive. In addition, these effects
combined with the small (extrinsic) value of the prize relative to the
median income of the participants, the long odds of winning the prize,
the lack of differences in participation among professions, and the
foreseeable additional costs of winning may suggest that (3) the effect
of a prize competition on participation goes beyond the actual value of
the prize itself, suggesting workers have, in fact, internalized some of
the benefits of participating in an organizational task.

We also fined that, although the WPLACE and PCARE solicitation
treatments are equally effective on average, responses appear sensitive
to the gender of the solicited person. Women's participation is greater
when emphasizing the patient care whereas men's participation is
significantly lower, controlling for the profession and position inside
the organization. This finding suggests that gender may be an important
factor influencing sensitivity of responses to solicitations concerning
the organizational mission.

At the same time, only an insignificant gender-based differences with
respect to participation in the PRIZE solicitation treatment was found:
women's participation was slightly higher but not significant than
men's, all else being equal. This evidence indicates that gender
differences in preferences, such as competitive inclinations or risk
aversion, may not exert great influence on responses of workers to
contests inside organizations.

We believe these results have three main implications for comparable
organizations and, more broadly, the internal provision of public goods.

The first implication is that announcing a competition for an individual
prize foster workers' participation in organizational tasks beyond the
value of the awarded prize itself. That is, prizes generate two opposing
externalities that help workers internalize the public good effects of
their organizational contributions. This result is important because it
highlights a relatively less understood function of contests that is to
mitigate the free riding incentives on organizational tasks.

A second implication is that offering the opportunity to lead collective
projects can exacerbate the free riding incentives. This result may
appear contrary to intuition. In theory, one may benefit more from
leading a project than winning an iPad. For instance, one may use the
opportunity to signal project management skills to the management aiming
for a career advancement; or steer some of the resources towards assets
or problems that are relatively more beneficial to his or her situation
compared to the rest of the organization. If so, why a negative result?
We believe that the private benefits from winning were negligible in our
setting. First, the opportunity for a career advancement is small
because medical staff gets promoted on the basis of other parameters
(e.g., the quality of care provided). Second, the peer evaluation and
vetting by the management ensure that the winning contributions yield as
distributed benefits as possible. These aspects may have eliminated the
possible private benefits from leading a project, resulting in poor
participation rates. This result is important because projects need to
be lead by someone and making more resources available does not seem to
increase volunteers.

A third implication is that participation in organizational tasks is
sometimes triggered by mission-based preferences. Although we find
evidence that these preferences can be an effective incentive, we also
find gender-based selection effects that are difficult to predict
ex-ante. Our experiment does not provide any insights to better
interpret these differences. But a large literature has investigated
gender-based difference in preferences \citep[see][ for a
review]{croson2009gender} or difference in self-stereotypes
\citep{coffman2014evidence} that could explain some of these effects.
Yet, more experimentation is needed to understand the different drivers
in the field inside organizations.

A few limitations of this study deserve consideration. The first is that
the validity of our causal interpretation of the results rests on a few
conventional assumptions \citep{rubin1974estimating}. These include the
``no interference between units'' assumption. In our study, it is
possible that communication among staff assigned to different treatment
arms could have influenced decisions to participate. The magnitude of
this interference would depend on intensity of staff communication and
the density of social interactions. Both of which should be small
because (i) an individual competition may provide only weak incentives
for information sharing and (ii) the staff members are scattered across
multiple buildings on the hospital campus. Even so, a potential
inference bias may alter the results towards a null effect as
differences in employee participation should converge towards zero when
communication spreads the content of the different email solicitations.
This goes against our results. Moreover, by looking at the temporal
dynamics of submissions, we find no indication of a convergence in the
participation rates. Hence, the assumption of no interference seems
appropriate.

Another potential limitation is that staff members may have left the
solicitation email that was sent to them unopened or unread, thus
non-complying with the assigned solicitation treatment. As this kind of
noncompliance is almost entirely unobserved,\footnote{The email was sent
  using the internal messaging system of the Heart Center, which, at the
  time, was not collecting individual analytics.} the analysis follows
an \emph{Intention-To-Treat} (ITT) approach, discarding entirely any
information about the solicitation treatment actually received. The main
drawback of an ITT analysis is that it does not answer questions about
causal effects of the content of the solicitation itself, only about
causal effects of the assignment to a solicitation treatment.

Finally, our results have implications that extend beyond the specific
organization under study. While the choice of focusing on health care
workers may limit the generalizability of our results in some respect,
it should be noted that in the US alone health care spending accounts
for 17 percent of the GDP (in 2015). And, more generally, our study
results are also directly applicable to a variety of other professions
exposed to a public good dilemma (e.g., teachers, public servants,
researchers). In all these settings, our study suggests that contests
soliciting employee contributions and awarding an individual prize to
the winning contribution appear an effective way to foster the internal
provision of public goods inside organizations.

\renewcommand\refname{References}
\bibliography{refs.bib}

\end{document}