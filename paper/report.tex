\documentclass[11pt, titlepage]{article}
\usepackage[utf8]{inputenc}
\usepackage{amssymb}
\usepackage{amsmath}
\usepackage{amsfonts}
\usepackage{indentfirst} % Indent paragraph
\usepackage[norule,bottom]{footmisc} % Footing options
%\usepackage[justification=centering,textfont={sc},labelfont={rm}]{caption}

% Page settings
\usepackage[left=1.5in, right=1.5in, top=1in, bottom=1in]{geometry}
\usepackage{setspace}
\onehalfspacing
%\doublespacing  % \singlespacing 

% Font
\usepackage{lmodern}
	% times, palatino, lmodern, tgtermes
	% bookman, charter, tgschola, pslatex
%\renewcommand{\familydefault}{tgbonum}

% Tools
\usepackage{beamerarticle} % Not sure
\usepackage{todonotes} % Todos
\usepackage{appendix} % Appendix
\usepackage{array,booktabs,longtable,rotating} % Tables
\usepackage{siunitx} % Align decimal points within tables
%\usepackage{lineno}+ %\linenumbers

% Sections, captions, etc.
\usepackage{sectsty} % Section styles
\sectionfont{\centering\scshape} % \normalfont
\subsectionfont{\centering\scshape}
\usepackage{titlesec}
\titlelabel{\thetitle.\quad}
%\titleformat{\subsubsection}[runin]{\em}{\thesubsubsection}{1em}{}
\titleformat{\subsubsection}[runin]{\em}{}{1em}{}


% Links
\usepackage{hyperref}
\hypersetup{%
  draft
%  colorlinks=false,% hyperlinks will be black
%  linkbordercolor=false,% hyperlink borders will be red
%  pdfborderstyle={/S/U/W 1}% border style will be underline of width 1pt
}

% Position tables {here, top, bottom, page}
\makeatletter
\def\fps@table{htbp}
\makeatother

%% ... at the end of paper
\usepackage{endfloat} % if needed, check package "float"

% Create new minipage environment for notes 
% at the bottom of tables or figures
\newenvironment{tablenotes}[1][Note:]{
  \vskip 1.8ex
  \begin{minipage}{\textwidth}\itshape\footnotesize{#1}
} {\end{minipage}}


% Graphics
\usepackage{graphicx,grffile}
\makeatletter
\def\maxwidth{\ifdim\Gin@nat@width>\linewidth\linewidth\else\Gin@nat@width\fi}
\def\maxheight{\ifdim\Gin@nat@height>\textheight\textheight\else\Gin@nat@height\fi}
\makeatother
% Scale images if necessary, so that they will not overflow the page
% margins by default, and it is still possible to overwrite the defaults
% using explicit options in \includegraphics[width, height, ...]{}
\setkeys{Gin}{width=\maxwidth,height=\maxheight,keepaspectratio}
% set default figure placement to htbp
\makeatletter
\def\fps@figure{htbp}
\makeatother

\usepackage{natbib}% plainnat, abbrvnat
\bibliographystyle{plainnat}
\setcitestyle{authoryear,open={(},close={)}}
%\bibliographystyle{aer}


\setlength{\emergencystretch}{3em}  % prevent overfull lines
\providecommand{\tightlist}{%
  \setlength{\itemsep}{0pt}\setlength{\parskip}{0pt}}



\title{Incentives for Public Goods Inside Organizations: Field Experimental
Evidence\thanks{Blasco: Harvard Institute for Quantitative Social Science, Harvard
University, 1737 Cambridge Street, Cambridge, MA 02138 (email:
\href{mailto:ablasco@fas.harvard.edu}{\nolinkurl{ablasco@fas.harvard.edu}}).
Jung: Harvard Business School, Soldiers Field, Boston, MA 02163 (email:
\href{mailto:oliviajung@gmail.com}{\nolinkurl{oliviajung@gmail.com}}),
Lakhani: Harvard Business School, Soldiers Field, Boston, MA 02163, and
National Bureau of Economic Research (email:
\href{mailto:k@hbs.edu}{\nolinkurl{k@hbs.edu}}). Menietti: Harvard
Institute for Quantitative Social Science, Harvard University, 1737
Cambridge Street, Cambridge, MA 02138 (email:
\href{mailto:mmenietti@fas.harvard.edu}{\nolinkurl{mmenietti@fas.harvard.edu}}).
We gratefully acknowledge the financial support of the MacArthur
Foundation (Opening Governance Network), NASA Tournament Lab, and the
Harvard Business School Division of Faculty Research and Development.
This project would not have been possible without the support of Eric
Isselbacher, Julia Jackson, Maulik Majmudar and Perry Band from the
Massachusetts General Hospital's Healthcare Transformation Lab.}}
\author{Andrea Blasco \and Olivia S. Jung \and Karim R. Lakhani \and Michael Menietti}
\date{Last updated: 13 March, 2018}

\begin{document}
\maketitle
\begin{abstract}
We conduct a natural field experiment at a medical organization that
sought contributions to organizational improvement from its 1200
employees. Offering a prize for best submissions boosted participation
without affecting the quality of the submissions, as measured by peer
ratings and the management. By contrast, offering financial resources to
lead one's own project implementation had a negative effect on
participation with no effects on quality. Mission-oriented solicitations
towards improving the care of patients or the workplace were equally
effective on average, but responses were sensitive to the solicited
person's gender (with women responding more than men to mission-oriented
incentives). Implications xxxx.

\smallskip\noindent 
JEL Classification: D23; H41; M52.

\smallskip\noindent 
Keywords: innovation contest; free rider problem; social preferences; altruism; idea generation; organization of work.
\end{abstract}


\clearpage
\tableofcontents
\setcounter{tocdepth}{2}
\clearpage

\section{Introduction}\label{introduction}

Workers employed by firms are often expected to make contributions that
go beyond an effective and efficient execution of their own tasks, such
as to put their knowledge to use via suggestions for improvement
(Osterman 1994). Contributions of this type are important because
generate positive externalities for the other members of the
organization, but are also difficult to measure objectively so that it
is hard to fully compensate the worker for its contribution. This paper
is concerned with understanding what motivates employees to ``go the
extra mile'' at work by choosing to perform discretionary tasks for the
benefit of the organization, instead of behaving opportunistically; and
how to design work environments that foster this kind of contributions
inside organizations.

An extensive literature in economics points to volunteering time and
effort at work as a way to obtain extrinsic benefits, such as career
advancements or a gain in reputation.\footnote{Since career advancements
  are uncertain and depend on a variety of conditions and not on
  productivity alone, the prospect of a promotion provide motivations to
  do extra-work \citep{kreps1997intrinsic}. \citet{gibbons1999careers}
  gives a survey of the theoretical literature on careers in
  organizations.} But it is also possible to link this kind of behavior
to a form of \emph{altruism} that goes from the worker to the other
members of the organization
\citep{bandiera2005social, rotemberg2006altruism} and from the worker to
external people who might benefit from the activities of the
organization (e.g., its customers)
\citep{delfgaauw2005dedicated, delfgaauw2008incentives, prendergast2007motivation}.
These altruistic feelings that acts as a motivation can be also inspired
by the ``mission'' of the organization
\citep{akerlof2005identity, besley2005competition}, like providing
healthcare to patients in hospitals, education to students in schools,
services to citizens in government agencies. However, the lack of
systematic empirical evidence on employee behavior makes difficult to
assess the strength of these different motivations, thus giving no
prediction as to which direction the dilemma faced by workers inside
organizations might be affected by them.\footnote{There is a large
  literature that has looked at these preferences in the laboratory in
  settings that mimic the employee-employer relationship {[}see XXX for
  example{]}.}

In this study, we conduct a natural field experiment to provide evidence
on the effectiveness of personal rewards and mission-oriented incentives
within an internal innovation contest at a medical organization. The
internal innovation contest aimed to spur voluntary contributions
(project proposals) to organizational improvement from its more than
1200 employees. The experiment tested incentives for contributing by
manipulating the content of personalized emails soliciting staff
participation, such as offering (\(i\)) awards for winning submissions,
(\(ii\)) funding for implementation of own projects, and appealing to
intrinsic motivations towards (\(iii\)) improving patient care (a
mission-oriented incentive) or (\(iv\)) the workplace.

The focus on the health care delivery context is important for two main
reasons. First, the need for organizational improvement and innovation
is vastly noted \citep{cutler2012reducing}. Second, health care
professionals are commonly seen as willing to step beyond the boundaries
of their contractual duties to offer better care
\citep{delfgaauw2005dedicated}, which makes the comparison of different
incentives towards a public good especially relevant and interesting.

The focus on employee behavior within an innovation contest follows both
practical and theoretical considerations. From a practical standpoint,
the widespread adoption of internet-based centralized information
systems has made organization-wide contests easy to design and run so
that they are increasingly common in most organizations.\footnote{Among
  the many examples of internal contests that have appeared in the news
  are the Apple's 2016 contest among its store employees seeking ideas
  on how to improve the way it sells iPhones (``Apple seeks `pie in the
  sky' ideas for innovation,'' Computerworld, 2013); Xerox's internal
  contest seeking employees ideas on how to make a more environmentally
  friendly workplace environment (``Xerox employees green ideas save
  company \$10.2 million,'' The Guardian, 2010); and AT\&T's ideation
  contests seeking employee ideas about new products (``AT\&T develops
  employee ideas for innovation,'' The Wall Street Journal, 2014).} This
makes relatively inexpensive for researchers to intervene by altering
the communications around the contest in a way to separately identify
channels associated with extrinsic and intrinsic reasons for
participating. For example, a personalized messaging strategy may either
convey information about the presence of personal awards for the
winners, or omit that information. An internal contest offers thus a
very convenient setting for experimental interventions to study how
employees self-select themselves into new tasks without affecting the
normal operations of the firm.

From a theoretical perspective, the analysis of employee voluntary
contributions to a public good in a competitive setting is very
interesting since much of the empirical work on voluntary contributions
and on contests has been done separately. However, theoretical work in
the context of public good games has shown that prizes generate two
opposing externalities : xxx and xxx. These externalities can mitigate
the free-riding problem, thus increasing contributions to the public
goods. We extend this reasoning to inside the organizations and we use
our experimental conditions to see whether prizes can be seen as a way
to internalize xxxx.

We evaluate the causal impact of the randomly assigned experimental
conditions on two main response variables: (a) employee participation
measured by the decision to submit a proposal and engage in an
organizational improvement task and (b) the quality of the submissions
as measured by (over 12,000) peer ratings and by assessments of the
management.

Testing the presence of both participation and quality effects is
important as the presence of systematic quality differences associated
with different motivations would substantially complicate the problem of
incentives for the organization (a higher employee participation is a
less desirable goal if it is driven by low-quality proposals). We also
investigate the presence of sorting effects based on the gender,
profession, and position inside the organization of the solicited
employee, and whether there are heterogenous treatment effects
associated with these characteristics. For example, following the
literature on gender-based differences in preferences
\citep{croson2004gender}, the gender may impact the willingness to
participate in a contest for prizes, complicating the analysis of the
incentives.

We find that the opportunity of winning a prize dominates all other
incentives, leading to a sizable and significant increase in the rate of
employees submitting project proposals. We also provide evidence that
such an increase in submission rates is without lowering the quality of
submissions and may go beyond the extrinsic value of the prize,
consistently with our theory of prizes as means to internalize public
goods.

We also find that, on the contrary, offering grant money for leading
implementation of one's own submitted project proposal is the least
effective incentive, leading to a lower employee participation to the
innovation contest relative to all other solicitation treatments with no
difference in quality.

By looking at the sorting patterns by gender, profession, and position
inside the organization, we find that mission-oriented incentives result
in responses that appear sensitive to the gender of the solicited
person. In particular, women's response to solicitations for improving
patient care is higher than men's. We also find that women's and men's
response to solicitations for prizes are the same; thus suggesting that
gender differences in preferences, such as competitive inclinations or
risk aversion, may not exert great influence on responses of workers to
the competition-for-prizes incentive.

Overall, our findings imply that employee participation is different
under different solicitation treatments, but with small and
insignificant effects on the quality of the proposals. This means that
understanding what motivates workers to self-select into the task is
important. Our results seem to suggest that workers better internalize
the positive effect of their contributions and participate more only
when they are awarded a prize, even if the value of the prize is small
relative to the costs associated with winning the competition and being
selected to carry out implementation work for their proposal. {[}xxxx{]}

\section{Literature}\label{literature}

Our work adds to the literature that analysis the problem of free-riding
inside organizations, e.g., Alchian and Demsetz (xxxx). Much of xxx,
however, is focused on compensation schemes for team production, such as
internal contests (e.g., Erev et al., 1993), incentives based on group
performance (Holmstrom and Milgrom, 1990), specialized task structures
(Itoh et al., 1991), and reputation and peer monitoring (Che and Yoo,
2001). But the focus in xxx is more on team incentives, where peer
monitoring could solve. In our study, the mechanism of peer monitoring
is prevented by the innovation contest xxxx. We add to these literature
by focusing on xxx outside of teams.

This work adds to the empirical research on the role of prizes in the
workplace. Existing studies, however, have focused on contests where xxx
xxx and contest organizers xxxx. Much less attention has been given to
contests where output benefits contestants and competitors are affecting
one another other than via competition for prizes. There exists
literature on prize-mechanisms for public goods, but these have not been
studied inside organizations. Less studies have looked at the winning of
grants. And framing? But not in terms of the public goods. There exists
an extensive literature in economics on the use of contests as a source
of incentives inside firms {[}xxxx{]}. Much of the existing theoretical
literature, however, presumes that agents are motivated to compete on
the basis of the utility derived by winning personal awards. Less
attention has been devoted to situations in which a competitor's
performance generates public good effects for the other competitors, and
the whole organization (a notable exception is XXX). By focusing on
incentives to workers to do organizational tasks (tasks with public good
effects for the organization), we improve the existing literature in
this direction.

Economists have long recognized that prize-based competitions are an
important source of incentives inside organizations
\citep{lazear1981rank, green1983comparison, nalebuff1983prizes, mary1984economic}.
Much of the existing theoretical literature in labor economics, however,
presumes that agents are motivated to compete solely on the basis of the
utility derived by winning one of the prizes. Less attention has been
devoted to situations in which a competitor's performance generates
public good effects for the other competitors, and the whole
organization; a notable exception is XXX. By focusing on incentives to
workers to do organizational tasks (tasks with public good effects for
the organization), we improve the existing literature in this direction.

for which there exist consistent findings across many different
empirical settings, including sport competitions
\citep{ehrenberg1990tournaments}, production competitions in firms
\citep{knoeber1994testing, terwiesch2008innovation}, and more recently
online competitions
\citep{boudreau2011incentives, boudreau2016performance}.

To be sure, free riding incentives inside organizations have been widely
studied in labor economics, especially in the context of team production
\citep{erev1993constructive, hamilton2003team, boning2007opportunity, gibbs2014field}.
However, our study differs from much of the existing literature in that
it focuses on an individual competition where the team component is
missing. That is, the public good dilemma comes from externalities
towards anyone in the organization, not just a set of identified team
members. It follows that one can remove from consideration conventional
team dynamics such as peer pressure, monitoring, reciprocity among team
members, and other kinds of social interactions that have been shown to
affect behavior in the presence of free riding incentives.

Our study is also related the literature in public economics that
studies prize-based mechanisms to foster the provision of public goods.
\citet{morgan2000funding} appears to be the first to note that
fixed-prize lotteries -- a special case also known as ``Tullock''
contest -- are widely used tools among non-profit fund raising firms,
showing conditions under which these may increase the provision relative
to voluntary contributions. This insight has spurred much attention in
public economics with several studies testing this idea empirically
\citep[see][ for a survey]{vesterlund2012voluntary}. However, as noted
by Vesterlund, the existing evidence on the profitability of lotteries
for charities is only mixed. Our work extends the existing literature on
the topic by focusing on an organizational setting where monetary
contributions are replaced by effort and the ``greater good'' is helping
the organization achieve its goals. Within this context, we find
evidence that fixed-prize contests are a profitable tool to foster
public good effects inside firms.

Finally, our work provides support to the incentive effect of
mission-oriented preferences -- inner satisfaction from helping the
organization achieve its goals --
\citep{akerlof2005identity, besley2005competition, delfgaauw2005dedicated, delfgaauw2008incentives, prendergast2007motivation, rotemberg2006altruism}
and social preferences at work
\citep{bandiera2005social, bandiera2008social, bandiera2013team, dellavigna2016estimating}.
According to this perspective, workers are motivated agents. They do
their work because they care about their co-workers, employers, and
customers. Theoretical models suggest different ways in which managers
can exploit these intrinsic motivations to raise individual levels of
participation and productivity. Here, we use announcing an internal
contest for organizational improvements to make these motivations
salient. We find that emphasizing mission-oriented motivations has
countervailing effects: positive for women and negative for men. While
this finding is consistent with altruism being an important driver of
effort inside organization, it also suggests that people are sensitive
to the framing and in ways that may be difficult to predict ex-ante.

\section{Conceptual framework and
predictions}\label{conceptual-framework-and-predictions}

In this section, we conceptualize an internal solicitation for
innovation project proposals to improve the operations of the
organization as a voluntary contribution mechanism for a public good.
Successful proposals are viewed as non-excludable because innovation
leads to improvements for everyone in the workplace (including customers
by increasing the quality and efficiency of the services provided).
Submitting a proposal requires costly effort by employees, such as the
time to identify a problem, form a proposal, write up a concise
description, and the potential for further involvement during proposal
implementation.

Consider the public good \(Y\) constitutes the sum of innovation
projects to improve the organization.\footnote{Instead of using a
  summation, we could have used different functional forms for the
  collective benefits (e.g., the max). However, the presence of
  free-riding incentives does not crucially depend on this specific
  assumption.} Imagine that the quality of each project is randomly
drawn from a discrete distribution, the same for every contributor
(every employee who contributes is assumed to be equally likely to come
up with a useful idea). Each proposal can be of high quality with
probability \(\nu\) and of low quality with probability \(1-\nu\). If a
proposal is of low quality, then the value for the organization is
normalized to zero. The quality of proposals is learned only after the
agent paid the cost of effort.

Let consider first the simplest case where the probability \(\nu=1\) so
any project is of high quality for sure. We assume a linear model of the
utility of a typical employee who contributes \(x\) and benefits from
total contributions of \(Y=\sum x\):

\begin{equation}\label{eq:utility}
  u(R,~ Y) =  \gamma Y + \delta x + \frac{x}{Y} R - c x.
\end{equation}

The benefits of contributing derive from three sources. First, there is
an altruistic benefit from the improved workplace, \(\gamma Y\). The
altruistic benefits are the crux of public goods. Only the existence of
an improved workplace is desired and the source of contributions is
irrelevant. Thus, everyone would prefer to free ride on others' efforts.
Second, participants have some chance of winning the contest and can
expect to derive benefits from the prizes, \(\frac{x}{Y} R\), where, for
simplicity, all efforts have an equal chance of being selected as the
winner, as in \citet{morgan2000financing}. The personal reward \(R\) can
be thought of as a pecuniary prize, but it could also be an increase in
prestige or recognition or any combination of the above. Finally,
employees may have an egotistic motivation for contributing ``per se,''
regardless of winning and the effect on others, which is captured by
\(\delta x\). This includes the case in which workers may derive a
personal satisfaction from contributing personally to the organization,
often called warm glow preferences for giving \citep{andreoni1995warm}.
Since we cannot observe the distinction between altruistic and warm-glow
motives in our empirical setup, we are going to impose later that these
preferences are such that \(\delta=0\).

Contributors incur some cost from developing and submitting a proposal,
\(c x\). If there are \(n\) employees the public goods dilemma arises
when \(\gamma+\delta < c < n\gamma+\delta\). Then no individual would
contribute without a reward as costs exceed individual benefits, but
everyone would be better off if everyone contributes.

Suppose contributing a proposal is a discrete choice by employees. An
employee can either contribute a single proposal \(x=1\) and receive
utility of

\begin{equation}
    u_1 = \gamma \hat Y + \delta + \sum_{k=1}^{n}\Pr(Y=k)\frac{R}{k}  - c, 
\end{equation}

where \(\hat Y\) denotes the expected level of contributions and
\(\Pr(Y=k)\) is the probability of having \(k\) total contributions. Or
they can contribute nothing \(x=0\) and receive utility of

\begin{equation}
  u_0 = \gamma (\hat Y - 1).
\end{equation}

If there are \(n\) employees, then the unique symmetric mixed-strategy
equilibrium is for each employee to contribute a proposal with
probability \(p>0\). After using the binomial probability for
\(\Pr(Y=k)\), the payoff-equating condition to find a mixed-strategy
equilibrium is:

\begin{equation} \label{eq: mixed-strategy}
  \frac{1- (1-p)^{n}}{n p} = (c- \gamma - \delta) / R.
\end{equation}

This equation admits one single solution \(p^*\) which cannot be
expressed explicitly. Using a first order Taylor expansion around \(p\),
the equilibrium probability can be approximated as follows:

\begin{equation} \label{eq: probability}
  p^*  \approx \frac{2 (R- c+\gamma +\delta )}{(n-1) R}. 
\end{equation}

The analysis of the above model is used to derive the following
predictions.

\begin{enumerate}
\def\labelenumi{\arabic{enumi})}
\item
  The probability of contributing a proposal to improving the
  organization is zero when the prize for winning is sufficiently small
  relative to the individual cost of effort minus the preference for the
  public good (i.e., \(R< c-\gamma +\delta\)).
\item
  The probability of contributing a proposal to improve the organization
  increases with the value of the prize for winning.
\item
  The probability of contributing a proposal to improve the organization
  increases with the extent of individual preference for the public good
  (\(\gamma+\delta\)).
\end{enumerate}

Now suppose that the probability \(\nu\) is less than one. The
equilibrium public good \(Y\) is not deterministic but follows a
binomial distribution with average \(E[Y] = p^{**}\nu n\), where the
equilibrium probability \(p^{**}\) can be derived as before with the
only difference being that it is also an increasing function of the
probability \(\nu\). This leads to the following prediction.

\begin{enumerate}
\def\labelenumi{\arabic{enumi})}
\setcounter{enumi}{3}
\tightlist
\item
  If the public good depends on the quality of each contribution and
  every agent is equally likely to make a proposal of high quality, then
  the higher the probability of contributing, the higher is the average
  public good.
\end{enumerate}

This framework can be extended to the case of individuals with
heterogeneous costs. In the appendix, we explicitly consider the case of
two types of individuals with different marginal costs of effort that
form two groups of equal size. The symmetric mixed-strategy equilibrium
is then characterized by the vector of probabilities of contributing
with a proposal \((p_1^\star, p_2^\star)\). Here, the analysis of the
payoff-equating conditions for the mixed-strategy equilibrium shows that
the higher the marginal cost of effort minus preference for
contributing, the lower the equilibrium probability of individuals
(i.e., \(p_1^\star > p_2^\star\) when \(c_1 < c_2\), and vice versa).
This leads the final prediction.

\begin{enumerate}
\def\labelenumi{\arabic{enumi})}
\setcounter{enumi}{4}
\tightlist
\item
  If individuals have heterogeneous costs, then the probability of
  contributing a proposal to improve the organization is higher for
  agents with lower costs (positive sorting).
\end{enumerate}

\section{Context, experimental design,
data}\label{context-experimental-design-data}

\subsection{Context}\label{context}

The medical organization in which the experiment was carried out is the
Massachusetts General Hospital's (MGH) Corrigan Minehan Heart Center, or
the ``Heart Center'' for short. Founded more than a hundred years ago,
the Heart Center is a prominent medical organization in the United
States and a teaching hospital of the Harvard Medical School. It serves
thousands of patients every year, occupies more than 35,000 square feet
of office space, and employs more than 1,200 people (nurses, physicians,
researchers, technicians, and administrative staff) scattered across
several buildings on the MGH's main campus in downtown Boston and a few
other satellite locations.

The opportunity for this study was the Heart Center's launch of the
Healthcare Transformation Lab (HTL), an initiative aimed at developing
innovative healthcare process improvements to enhance the healthcare
safety and delivery of the hospital.\footnote{See the HTL's web site at:
  www.healthcaretransformation.org} The launch of the HTL was
accompanied by the announcement of an internal innovation contest,
called the Ether Dome Challenge,\footnote{The name is taken from a
  historical place on MGH's main campus where the first public surgery
  using anesthesia was demonstrated in 1846.{]}} that sought to engage
all staff members to participate.

The communication around the innovation contest highlighted the
opportunity for staff to help in the selection process of the ideas and
a commitment by the Heart Center Management that the leading ideas would
be provided appropriate resources so that they could be implemented. The
announcement on the contest website reads:

\begin{quote}
``If you've noticed something about patient experience, employee
satisfaction, workplace efficiency, or anything that could be improved;
if you've had an inspiration about a new way to safeguard health; or if
you simply have a cost-saving idea, then now is the time to share your
idea.''
\end{quote}

Within this context, we worked with the HTL administrators to design the
innovation contest and the experimental treatments that were then
randomly assigned to all staff members of the Heart Center in order to
identify and compare the effect of different motivations on employee
participation in the contest.

\begin{figure}
\centering
\caption{Timeline of the innovation contest}
\label{timeline}
\includegraphics{../figs/timeline.pdf}
\end{figure}

The innovation contest can be divided into three main phases (Figure
\ref{timeline}). The first is a four-week submission phase in which all
staff members are encouraged to identify one or more organizational
problems and submit proposals addressing them. Employee participation is
voluntary and all project submissions can be done online via the website
of the contest. There is no limit in the number of project proposals to
submit and proposals could cover any issue within the organization. The
only constraints are: (1) each proposal is limited to approximately 300
words to lower the costs of entry and encourage broader participation;
and (2) team submissions are not permitted. The second constraint is to
ensure that randomly assigned treatment effects, which we describe next,
could be isolated, identified, and matched to individual participants.
Another advantage of the individual submission constraint is that it
lowers the incentives to communicate or exchange information among
employees by preventing groups to form, thus lowering the risk of
``interference'' among individuals in the different treatments, a
problem we will discuss later. In addition, the website provides no
feedback information about the status of the contest during the
submission period, so that individual decisions could not be easily
influenced by the perceived popularity of the contest or previous
submissions.

The second is a two-week project evaluation phase in which all staff
members are invited to rate the merit and potential of submitted
proposals on a five-point rating scale. Such evaluations are done on the
website of the contest where each evaluator is shown a list of
anonymized proposals to read and rate. Proposals are presented in
batches of 10 each and in random order to ensure an even exposure. Each
proposal is described by a title, a main description of the problem to
solve, and the proposal. Voting is then introduced by the following
text: ``Rate this idea'' followed by the rating scale: 1-low; 2; 3; 4;
5-high. Evaluators can decide how many proposals to rate and they get a
chance to win a limited edition T-shirt as a compensation for their
efforts. Ratings would be kept confidential. And, as in the submission
phase, the website provides no feedback or any other kind of information
that would influence individual judgment until the evaluation phase is
over.

The third is an implementation phase in which employees having submitted
proposals that would have been highly rated by peers and judged as
particularly promising by the HTL staff are invited to submit a full
proposal detailing plans for implementation. Following evaluation by MGH
senior leadership, top proposals are then selected to receive support
and funding for implementation. This final phase takes a few months to
complete, essentially the time necessary to select and implement the
best projects.

\subsection{Experimental design}\label{experimental-design}

Our experiment was conducted during the normal operations of the Heart
Center. The basic idea of the experiment is to randomize the content of
the communication announcing the innovation contest to all staff
members. The start of the submission phase was indeed announced to
everyone in a series of personalized emails. A direct message was sent
to each contact in the list of employees' emails from our subject pool.
The content of this communication with a placeholder for our
solicitation treatment is reported below:\footnote{An image of the exact
  email is in the Appendix.}

\begin{quote}
Dear Heart Center team member,

\textbf{Submit your ideas to {[}TREATMENT HERE{]}}

The Ether Dome Challenge is your chance to submit ideas on how to
improve the MGH Corrigan Minehan Heart Center, patient care and
satisfaction, workplace efficiency and cost. All Heart Center Staff are
eligible to submit ideas online. We encourage you to submit as many
ideas as you have: no ideas are too big or too small!

Submissions will be reviewed and judged in two rounds, first by the
Heart Center staff via crowd-voting, and then by an expert panel.
Winning ideas will be eligible for project implementation funding in the
Fall of 2014!
\end{quote}

The first paragraph of the above message was randomized into \emph{four}
different solicitation treatments, thus creating as many treatment
groups of equal size. The first solicitation (PRIZE) announced the
innovation contest as an opportunity to win individual awards (iPad
mini's) for top submissions. The second solicitation (FUND) announced
the opportunity for participants to win a \$20,000 budget for developing
their own project proposals. The other solicitations announced the
contest as an opportunity to improve the health care of their patients
(PCARE) or the workplace (WPLACE), without mentioning any particular
personal award for the winners (neither a prize nor the opportunity to
lead the implementation of own projects). In Table
\ref{experimental-design}, we report the exact words and randomly
assigned employees for each solicitation treatment.

\begin{table}
\centering
\caption{Experimental design}
\label{experimental-design}
\begin{tabular}{@{}lp{5cm}>{\raggedright}rr}
  \\[-1.8ex]\hline \hline \\[-1.8ex]
 & \multicolumn{1}{c}{\emph{Solicitation treatment:}}
						& \multicolumn{2}{c}{\emph{Employees:}}\\
						\cmidrule(lr){2-2}\cmidrule(lr){3-4} & 	 & freq. & \% \\ 
  \hline \\[-1.86ex]
PRIZE & Submit your ideas to win an Apple iPad mini & 312 & 25 \\ 
  [1.8ex] FUND & Submit your ideas to win project funding up to \$20,000 
			to turn your ideas into actions & 308 & 25 \\ 
  [1.8ex] PCARE & Submit your ideas to improve patient care at the Heart Center & 310 & 25 \\ 
  [1.8ex] WPLACE & Submit your ideas to improve the workplace at the Heart Center & 307 & 25 \\ 
  [1.8ex] Total &  & 1237 & 100 \\ 
   \\[-1.8ex]\hline \hline \\[-1.8ex]
\end{tabular}
\end{table}


A sample size of more than 300 units for each treatment group ensured a
sufficiently high statistical power based upon standard power
calculations on the difference of proportions. In testing the difference
of proportions between any two treatments, the probability of type-I
errors was slightly below \(0.80\) for \emph{small} differences at 5
percent significance level but higher than \(0.80\) for \emph{medium}
and \emph{large} differences at the more stringent 1 percent
significance level.\footnote{The definition of small, medium and large
  differences is given by \citet{cohen1992power}; e.g., a difference of
  5 percentage points of the pair \((0.05, 0.10)\) is considered a small
  effect: see \citet{cohen1992power} p.~158.}

Also, note the lack of a traditional ``control'' treatment in this
study. Since the experiment was run in a workplace, we were constrained
to carry out treatments having equal chances of being successful. This
prevented us from having a `null' treatment with no personalized
incentives messaging as a control group. Indeed, the analysis focused on
multiple comparisons of several unordered discrete treatments (e.g.,
prizes vs funding vs framing).\footnote{Nevertheless, if we were to
  think of one treatment as the benchmark against which to compare the
  others, the FUND treatment would be our best candidate because that is
  the default option for announcing internal grant programs and was part
  of the HTL's initial design before our cooperation in the experiment.}

These solicitation messages were sent three times: at the launch of the
submission phase, eight days from the launch and two days before the end
of the submission phase of the challenge.

The website of the innovation contest had supporting information about
the available prizes, funding, and timing of the initiative. The website
also required an institutional email address to login. Using this
feature, we designed the website graphics and layout to reinforce the
effect of the announcement: the headings, background images, a short
video, and the space just below a ``submit your ideas'' button were
designed to show the exact same first paragraph of the solicitation that
the employee received by email (i.e., text in Table
\ref{experimental-design}).

The MGH management and the HTL staff members were blind to group
assignment, which prevented potential bias in the communication of the
innovation contest that was not under our direct control. We also made
an effort to create a ``safe'' environment for employees submitting
proposals by making clear (in the application form) that the identity of
the proponents was going to be kept private unless the employee
self-identified, so that management could not identify workers without
their consent.

Finally, we relied only on official channels for communication to
strengthen the effect of the announcement and signal legitimacy of the
contest. Each employee received the same exact solicitation email three
times: at the launch, eight days from the launch and two days before the
end of the submission phase of the challenge. Starting from the second
week of the submission phase, information booths, flyers, and posters
were used to encourage everyone to take part in the event and respond to
the email solicitation. These flyers and posters were based on a
generic, undifferentiated version of the solicitation email without the
text of the treatments.

\subsection{Data}\label{data}

Our subject pool is the entire population working at the Heart Center as
of the end of 2014, a total of 1,237 individuals. For each individual,
we have administrative data on the gender, the type of profession, and
whether they had a fixed office location or not. Additional,
complementary data are available for a limited group of 378 employees
(31 percent) with self-reported demographics information, such as age
and years of tenure at the Heart Center, that are from an online survey
conducted about two months prior to the launch of the innovation
contest.

\begin{table}
\centering
\caption{Summary statistics by treatment}
\label{summary-statistics}
\begin{tabular}{@{}lccccccc}
  \\[-1.8ex]\hline \hline \\[-1.8ex]
 & \multicolumn{4}{c}{\emph{Assigned treatments:}} 
						& \multicolumn{2}{c}{\emph{All:}}\\
						\cmidrule(lr){2-5}\cmidrule(lr){6-7} & FUND & PCARE & WPLACE & PRIZE & \% & Obs. & P-value \\ 
  \hline \\[-1.86ex]
Other & 30 & 30 & 26 & 32 & 29 & 362 & 0.84 \\ 
  MD/Fellow & 19 & 18 & 18 & 18 & 18 & 226 &  \\ 
  Nursing & 51 & 52 & 56 & 51 & 52 & 649 &  \\ 
  Female & 69 & 70 & 75 & 75 & 72 & 890 & 0.16 \\ 
  Male & 31 & 30 & 24 & 26 & 28 & 347 &  \\ 
  [1.86ex] No office & 50 & 46 & 47 & 45 & 47 & 577 & 0.56 \\ 
  Office & 50 & 54 & 52 & 56 & 53 & 660 &  \\ 
   [1.86ex] Age* &&&&&&\\
18-25 & 6 & 8 & 8 & 6 & 6 & 24 & 1.00 \\ 
  26-35 & 29 & 29 & 31 & 26 & 29 & 107 &  \\ 
  36-45 & 18 & 19 & 24 & 16 & 22 & 81 &  \\ 
  $>$45 & 44 & 46 & 51 & 45 & 42 & 157 &  \\ 
   [1.86ex] Tenure* &&&&&&\\
$<$ 10 & 40 & 31 & 36 & 37 & 36 & 132 & 0.89 \\ 
  10-20 & 26 & 29 & 38 & 28 & 30 & 111 &  \\ 
  20-30 & 12 & 19 & 15 & 10 & 14 & 50 &  \\ 
  30-40 & 10 & 16 & 15 & 12 & 13 & 48 &  \\ 
  $>$40 & 10 & 4 & 8 & 8 & 8 & 28 &  \\ 
   \\[-1.8ex]\hline \hline \\[-1.8ex]
\end{tabular}
\begin{minipage}{\textwidth}\itshape\footnotesize
Note: This table reports the percentage of employees in our sample cross tabulated by the assigned treatment across the gender, profession, whether the employee had a fixed office location, age, and years of tenure at the Heart Center. For each categorical variable, the last column reports the p-value from a Pearson's Chi-squared test with the assigned treatment and the variable. The asterisk $^{\ast}$ indicates self-reported information obtained from an online survey polling employees about two months before the launch of the innovation contest.
\end{minipage}
\end{table}


We report summary statistics for the different variables (Table
\ref{summary-statistics}). A series of Chi-square tests of associations
between treatment groups and each separate variable fail to reject the
null hypothesis of independence; meaning that differences across groups
are not attributable to our experimental intervention. In other words,
treatment groups are statistically balanced.

The data also show that the large majority (72 percent) of employees in
our sample are women. This is due to the high fraction of workers being
nurses (52 percent) and the presence of a gender separation by
profession with nurses being predominantly women (92 percent).

As much of the clinical staff might be mobile, only half of the
employees (53 percent) have fixed office locations because they may be
on duty in multiple wards. The need of being mobile, however, is not the
same across professions. Nurses, for instance, are more likely to being
assigned to a ward than physicians or administrative workers, due to the
nature of their job. Within each profession, however, having a fixed
office location is usually correlated with experience and hierarchical
position inside the organization. Using our subset with tenure
information, a staff member with a fixed office location has 2 more
years of tenure in median compared to an employee assigned to a ward;
and the difference goes up to more than 3 years when we examine each
professional category separately.

Differences in income and education are other important differences
across professions. Though we do not have data on income, there exist
large differences in earnings across professions. According to the
United States Bureau of Labor Statistics, the median annual wage of a
physician was \$187,200 in 2015, which is about 60 percent higher than
the that of a registered nurse (\$67,490) and about 70 percent higher
than that of a laboratory technician (\$38,970). It follows that, if
staff members are motivated by the extrinsic value of the prize alone,
one should expect large differences in participation rates across
profession.

\input{../tables/summary.outcomes.tex}

Our main response variables include all the submissions to the contest,
participation in the evaluation phase, and the resulting ratings for
each project proposal as a measure of proposal quality. Basic summary
statistics (Table \ref{outcomes}) show the overall participation in the
submission phase is 5 percent of our sample with 60 submissions and a
total of 118 project proposals (an average of 2 project proposals per
submission).\footnote{Here, and throughout the paper, we exclude an
  additional 20 proposals from 11 employees who were not part of the
  Heart Center when the experiment was designed.} A 5 participation rate
is unsurprising when one considers a few key elements, such as the busy
environment, the fact that staff members have many conflicting
priorities, and the foreseeable additional costs of leading
implementation of a winning proposal. This result, however, does not
imply that the overall participation in the contest was modest. On the
contrary, when we combine data from the submission and evaluation
phases, the overall participation rate is 16 percent.\footnote{This is
  the union of employees who made submissions and evaluated proposals.}
This means that about 1 out of 5 staff members was engaged in one way or
another in the contest, indicating a good participation overall.

\section{Analysis}\label{analysis}

\subsection{Employee participation}\label{employee-participation}

We begin by focusing on the causal effect of our experimental
intervention on employee participation, which is defined by the
percentage of employees who made project submissions within the
four-week submission period of the contest.

\begin{figure}
\caption{Submission rates by solicitation treatment}
\label{fig: submit}
\includegraphics{../figs/part.bar.pdf}
\begin{tablenotes}
Submission rates for employees randomly assigned to different solicitation treatments (bars represent standard errors).
\end{tablenotes}
\end{figure}

Employees' submission rates range from 2 to 7 percent (Figure
\ref{fig: submit}) showing substantial variation across solicitation
treatments. A Fisher's Exact Test for Count Data fails to reject the
null of independence between this variation and the solicitation
treatments (p=0.026), thus indicating a statistically significant
association, which means that some of the observed variation in
submission rates is attributable to our experimental intervention.

Pairwise comparisons among individual solicitation treatments further
reveal that: (\emph{i}) employees in the PRIZE solicitation treatment
have 1.4, 1.6, and 3.2 times higher participation rates than those in
the WPLACE, PCARE, and FUND solicitation treatments, respectively;
(\emph{ii}) employees in the PCARE and WPLACE solicitation treatments
have basically identical participation rates; and (\emph{iii}) employees
in the FUND solicitation treatment have less than half of the
participation rates of the WPLACE and PCARE solicitation treatments.

\begin{table}
\centering
\caption{P-values for pairwise comparison of proportions}
\label{pairwise}
\begin{tabular}{@{}lccc}
  \\[-1.8ex]\hline \hline \\[-1.8ex]
 & FUND & PCARE & WPLACE \\ 
  \hline \\[-1.86ex]
PCARE & 0.124 &  &  \\ 
  WPLACE & 0.055 & 0.688 &  \\ 
  PRIZE & 0.003 & 0.132 & 0.269 \\ 
   \\[-1.8ex]\hline \hline \\[-1.8ex]
\end{tabular}
\begin{minipage}{\textwidth}\itshape\footnotesize
Note: This table reports the p-values of pairwise comparisons of proportions
						         among solicitation treatments.
\end{minipage}
\end{table}


To test to see whether these pairwise differences are statistically
significant, we use Pairwise comparison of proportions (Table
\ref{pairwise}). The analysis reveals that: a significant positive
difference in participation rates between the PRIZE and FUND
solicitation treatments (p=0.003); a marginally significant positive
difference between the PRIZE and PCARE solicitation treatments
(p=0.132); and an insignificant positive difference between the PRIZE
and WPLACE solicitation treatments (p=0.269); a significant negative
difference between the FUND and WPLACE solicitation treatments
(p=0.055); and a marginally significant negative difference between the
FUND and the PCARE solicitation treatments (p=0.124). Overall, these
findings are consistent with employee participation being higher under a
solicitation with personal awards incentives; and lower under a
solicitation with funding incentives.

A multiple linear regression model that explicitly controls for
observable differences across staff members serves to complement the
above univariate analysis. Let \(Y_i=1\) denote employee \(i\) making a
submission, and \(Y_i=0\) otherwise. We assume the conditional
probability of an employee making a submission is given by:

\[\Pr(Y_i=1) = \alpha_0 
                                    + \sum_{j} \alpha_{j} \text{SOLICIT}_{ij}
                                    + \text{JOB}_{i} 
                                    + \text{MALE}_{i} 
                                    + \text{OFFICE}_{i},\label{eq: submit}\]

where \(\alpha_0\) is a constant, \(\alpha_j\) is the causal effect of
the solicitation treatment \(j\) assigned to an employee \(i\)
(\(\text{SOLICIT}_{ij}\)), controlling for the employee's profession
(\(\text{JOB}_i\)), the gender (\(\text{MALE}_i\)), and a dummy for
office location (\(\text{OFFICE}_i\)) indicating whether the employee
had a permanent office instead of being assigned to a ward. As discussed
before, in our context, having a fixed office location is highly
correlated with the type of profession. Within each profession, however,
having a fixed office location is usually correlated with the experience
or hierarchical position inside the organization. Hence, more than just
the effect of having a fixed office location per se, this variable is
potentially controlling for income and hierarchical differences
occurring within each profession.

\begin{table}
\centering
\caption{Probability of submitting proposals}\label{participation ols}
\begin{tabular}{@{\extracolsep{5pt}}lccccc} 
\\[-1.8ex]\hline 
\hline \\[-1.8ex] 
 & \multicolumn{5}{c}{\textit{Dependent variable:}} \\ 
\cline{2-6} 
\\[-1.8ex] & \multicolumn{5}{c}{ $SUBMIT_{ij}=1$ } \\ 
\\[-1.8ex] & (1) & (2) & (3) & (4) & (5)\\ 
\hline \\[-1.8ex] 
 PRIZE & 2.53$^{**}$ & 2.53$^{**}$ & 2.52$^{**}$ & 2.46$^{**}$ & 2.45$^{**}$ \\ 
  & (1.21) & (1.21) & (1.21) & (1.21) & (1.21) \\ 
  & & & & & \\ 
 WPLACE & 0.37 & 0.37 & 0.35 & 0.38 & 0.30 \\ 
  & (1.09) & (1.09) & (1.10) & (1.09) & (1.10) \\ 
  & & & & & \\ 
 FUND & $-$2.57$^{***}$ & $-$2.57$^{***}$ & $-$2.55$^{***}$ & $-$2.49$^{***}$ & $-$2.38$^{***}$ \\ 
  & (0.86) & (0.86) & (0.85) & (0.86) & (0.85) \\ 
  & & & & & \\ 
 Job (nursing) &  & 0.14 &  &  & 1.85 \\ 
  &  & (0.82) &  &  & (1.23) \\ 
  & & & & & \\ 
 Job (MD) &  & $-$0.31 &  &  & $-$1.14 \\ 
  &  & (1.03) &  &  & (1.24) \\ 
  & & & & & \\ 
 Male (yes) &  &  & $-$0.54 &  & $-$0.42 \\ 
  &  &  & (1.33) &  & (1.64) \\ 
  & & & & & \\ 
 Office (yes) &  &  &  & 2.79$^{**}$ & 4.56$^{***}$ \\ 
  &  &  &  & (1.20) & (1.60) \\ 
  & & & & & \\ 
 Constant & 4.84$^{***}$ & 4.78$^{***}$ & 5.00$^{***}$ & 3.35$^{***}$ & 1.97 \\ 
  & (0.61) & (0.66) & (0.73) & (0.75) & (1.25) \\ 
  & & & & & \\ 
\hline \\[-1.8ex] 
Log Likelihood & -5545 & -5545 & -5545 & -5542 & -5540 \\ 
Observations & 1,237 & 1,237 & 1,237 & 1,237 & 1,237 \\ 
\hline 
\hline \\[-1.8ex] 
\end{tabular} 
\begin{minipage}{\textwidth}
\emph{Note:} This table reports OLS estimates with heteroskedasticity robust standard errors in parenthesis. All coefficients are multiplied by 100 to indicate a percentage point change in the probability of submitting. Solicitation treatment dummies are coded to indicate deviations from the overall probability of submitting. The asterisks $^{\ast\ast\ast}$, $^{\ast\ast}$, $^{\ast}$ indicate significance at 1, 5 and 10 percent level, respectively.
\end{minipage}
\end{table}


We report the OLS estimated coefficients for the above model (Table
\ref{participation ols}) expressed as solicitation treatment differences
relative to the overall mean participation. Results are consistent with
what discussed before. At the 95 level of statistical significance,
employees in the PRIZE solicitation treatment are about 2 percentage
points \emph{more} likely to submit compared to the overall mean,
whereas employees in the FUND solicitation treatment are 2 percentage
points \emph{less} likely to do so. Although these effects might appear
small in absolute terms, they are fairly large in comparison to the
overall participation rate (5 percent).

\begin{figure}
\caption{Sorting}
\label{fig: sorting}
\includegraphics{../figs/sorting.bar.pdf}
\begin{tablenotes}
Submission rates by the gender, job, and office status of the solicited employees (bars represent standard errors).
\end{tablenotes}
\end{figure}

\subsubsection{Sorting}\label{sorting}

We now look at how the willingness to participate in the contest is
associated with relevant individual characteristics, like the gender,
profession, and status inside the organization of the solicited worker.
Following the literature on gender-based differences in preferences
\citep{croson2009gender} --- like attitudes towards competitive settings
\citep{niederle2007women} --- one may expect gender to be a factor
driving participation in the contest, with men relatively more willing
to sort into the competition. Contrary to these expectations, the
submission rate for women (Figure \ref{fig: sorting}) appears higher
than that for men and testing the difference in a regression with
controls (as in the last column of Table \ref{participation ols}) gives
insignificant results (p=0.689). Similarly, one may expect the presence
of effects associated with differences in income and education as
captured by the profession of the solicited employee. Consistently with
this view, physicians appear to have a lower propensity to submit
(Figure \ref{fig: sorting}) relative to the other workers, whereas it
appears higher for nurses, but testing them in a regression with
controls gives again insignificant results (p=0.989). Finally, and
somewhat surprisingly, we find that having an office location, as
opposed to being assigned to a ward, is significantly (p=0.003)
associated with a higher submission rate, controlling for the other
characteristics. This evidence is suggesting that workers who occupy
higher positions inside the organization are also more willing to
contribute to common goods. Hence, and overall, we find no evidence
supporting sorting based on the gender and profession of the solicited
employee, whereas we find evidence for sorting based on having an office
location, which we interpret as a proxy for experience and status inside
the organization.

\begin{figure}
\caption{Submission time dynamics}
\label{dynamics}
\includegraphics{../figs/dynamics.pdf}
\begin{tablenotes}
The plot shows the evolution of the cumulative sum of submissions by solicitation treatment. Employee participation in the PRIZE treatment is higher than the other solicitation treatments at all periods. By contrast, employee participation in the FUND treatment is lower at all periods. The plot also shows little convergence of the participation rates over time.
\end{tablenotes}
\end{figure}

\subsubsection{Participation dynamics}\label{participation-dynamics}

We now turn to examining participation dynamics (Figure \ref{dynamics}).
Though our data may not allow for a complete analysis of participation
dynamics, looking at the overall submission patterns can be useful for
the following reason. If employees assigned to different solicitation
treatments were sharing (either face-to-face or electronically) the
content of their solicitation with others, one should expect
participation rates to converge over time, yielding estimates of the
causal effects of a solicitation treatment biased towards zero. Contrary
to these expectations, we find no evidence of convergence. Submissions
in the PRIZE solicitation treatment are constantly higher than in the
other treatments (except perhaps in the final week); at the same time,
submissions in the FUND treatment are constantly low. These patterns are
hence consistent with communication effects having little, or no,
consequences on our findings, a topic we will discuss in greater detail
later.

\begin{figure} 
\caption{Employee participation by gender or profession and solicitation treatment}
  \label{interactions}
  \centering
  \includegraphics{../figs/interactions.pdf}
\begin{tablenotes}
Panel (a) shows the proportion of submissions conditional on the gender of the solicited employee. Women seem more likely (about 5 percentage points) to participate than men in the PCARE solicitation treatment. Panel (b) shows the same proportion conditional on the profession of the solicited employee. Physicians, who are the highest paid professionals in our setting, are as likely to submit as any other worker in PRIZE solicitation treatment; thus suggesting little sorting based on income or other characteristics associated with a given profession.
\end{tablenotes}
\end{figure}

\subsubsection{Heterogeneous treatment
effects}\label{heterogeneous-treatment-effects}

Inspection of the participation rates conditional to the employee's
gender or profession (Figure \ref{interactions}) suggests relevant
treatment interactions based on the gender.\footnote{We find no
  significant differences for interactions with office location, which
  we do not report for space limitation.} Gender interactions may occur
as a result of three main factors: differences in risk taking, social
preferences (willingness to contribute to public goods), and competitive
inclinations. If women prefer to work on activities that are less risky,
more pro-social (e.g., aiming at improving people's health) and where
competition is less intense, then we should observe significant
treatment interactions. Similarly, one may expect treatment interactions
associated with the employee's profession to occur because, for example,
the prize opportunity could be relatively less effective for employees
with a higher income, such as doctors.

\begin{table}
\centering
\caption{Gender differences}\label{tab: probability submitting interactions}
\begin{tabular}{@{\extracolsep{5pt}}lccc} 
\\[-1.8ex]\hline 
\hline \\[-1.8ex] 
 & \multicolumn{3}{c}{\textit{Dependent variable:}} \\ 
\cline{2-4} 
\\[-1.8ex] & \multicolumn{3}{c}{ $SUBMIT_{ij}=1$ } \\ 
\\[-1.8ex] & (1) & (2) & (3)\\ 
\hline \\[-1.8ex] 
 PRIZE$\times$female & 2.99$^{*}$ & 2.95$^{*}$ & 2.84 \\ 
  & (1.68) & (1.79) & (1.78) \\ 
  & & & \\ 
 PCARE$\times$female & 1.25 & 1.21 & 1.08 \\ 
  & (1.57) & (1.61) & (1.61) \\ 
  & & & \\ 
 FUND$\times$female & $-$2.91$^{***}$ & $-$2.95$^{**}$ & $-$2.79$^{**}$ \\ 
  & (1.06) & (1.20) & (1.19) \\ 
  & & & \\ 
 WPLACE$\times$female & $-$0.49 & $-$0.52 & $-$0.62 \\ 
  & (1.35) & (1.44) & (1.43) \\ 
  & & & \\ 
 PRIZE$\times$male & 1.37 & 1.42 & 1.40 \\ 
  & (2.44) & (2.51) & (2.50) \\ 
  & & & \\ 
 PCARE$\times$male & $-$3.75$^{***}$ & $-$3.72$^{***}$ & $-$3.64$^{***}$ \\ 
  & (1.15) & (1.16) & (1.16) \\ 
  & & & \\ 
 FUND$\times$male & $-$1.67 & $-$1.65 & $-$1.48 \\ 
  & (1.70) & (1.65) & (1.66) \\ 
  & & & \\ 
 Constant & 4.80$^{***}$ & 4.79$^{***}$ & 1.87$^{*}$ \\ 
  & (0.69) & (0.70) & (1.10) \\ 
  & & & \\ 
\hline \\[-1.8ex] 
Job & no & yes & yes \\ 
Office & no & no & yes \\ 
Log Likelihood & -5542 & -5542 & -5538 \\ 
Observations & 1,237 & 1,237 & 1,237 \\ 
\hline 
\hline \\[-1.8ex] 
\end{tabular} 
\begin{minipage}{\textwidth}
\emph{Note:} This table reports OLS estimates with heteroskedasticity robust standard errors in parenthesis. All coefficients are multiplied by 100 to indicate the percentage point change in the probability of submitting. Solicitation treatment dummies are coded to indicate deviations from the overall probability of submitting. The asterisks $^{\ast\ast\ast}$, $^{\ast\ast}$, $^{\ast}$ indicate significance at 1, 5 and 10 percent level, respectively.
\end{minipage}
\end{table}


To isolate gender and profession effects, we employ a version of model
\eqref{eq: submit} with gender-treatment interactions. The regression
results (Table \ref{tab: probability submitting interactions}) show
similar results to the simple comparison of proportions. That is, after
gradually adding profession and office controls, interaction
coefficients remain stable across all specifications: the response of
men under the PCARE solicitation treatment is about 3 times the
magnitude and in the opposite direction of the women's response. By
subtracting these two coefficients, we find a significant difference
between men and women of about 5 percentage points (\(p=.018\)), which
is consistent with our previous analysis. Thus, and overall, men respond
less than women in the PCARE solicitation treatment, controlling for the
profession and office location. This effect could be due to gender
differences in preferences, as suggested by the literature, and we will
return on this topic in the discussion of the results.

\subsection{Employee participation in peer
evaluation}\label{employee-participation-in-peer-evaluation}

We now turn to examining the outcomes of the peer evaluation that
follows the submission period. In this phase, 113 project proposals
ended up being rated by a total of 178 employees (14 percent of our
sample) who volunteered for the task. Their effort yielded a total of
12,219 evaluator-proposal pairs, providing a very sensitive test for
differences in project quality across our solicitation treatments.

\input{../tables/ratings_by_treatment.tex}

We note (Table \ref{ratings}) that employees who are in the PCARE and
WPLACE solicitation treatments seem to have a higher propensity to
evaluate proposals than those in the PRIZE and FUND solicitation
treatments. Since evaluating proposals is another way for workers to
contribute to the organization, workers will have incentives similar to
those in the submission phase so that indirect treatment effects are
plausible. However, no evidence is found that the observed differences
in evaluation rates are attributable to our experimental intervention (a
Fisher's exact test gives a p-value of 0.339). In other words, our data
do not seem to support the hypothesis of treatment effects on evaluating
proposals.

We further check for sorting to see whether the evaluators are wholly
representative of the organization. Testing for the statistical
significance of coefficients for the profession, gender, and office
location in a linear regression on the probability of evaluating
proposals (results not shown), we find no differences based on the
employee's gender and profession albeit a significantly higher
participation from staff members with an office location. This evidence
is thus consistent with our previous results about participation in the
submission phase; suggesting the collected assessments of proposal
quality are made by a broadly representative sampling of opinions inside
the organization.

\subsection{Quality of the project
proposals}\label{quality-of-the-project-proposals}

The treatment interventions may not have only impacted the propensity to
make a submission, but the quality of the submission as well. Of
particular interest is any indication of a quantity versus quality
trade-off. For example, if the treatment which generated the fewest
submissions (FUND) also produced the highest quality submissions. A
quality versus quantity trade-off would increase the complexity of
choosing optimal incentives for employees. We examine the issue with the
assessments of quality made by peers in the evaluation phase of the
contest and, subsequently, by the management.

\subsubsection{Quality assessed by
peers}\label{quality-assessed-by-peers}

To check whether differences in the quality of the submissions can be
explained by the solicitation treatments of the submitter, we first look
at differences in the distribution of ratings obtained from peers.
Overall, a project proposal is given the ``neutral'' point (i.e., a
rating of 3) on a five-point scale about 30 percent of the times with
employees being more likely to give high (4-5) rather than low (1-2)
ratings. This pattern does not change much when we condition the data to
the solicitation treatment of the proponent (Figure \ref{fig: quality});
suggesting an equal distribution of good and bad quality projects across
the solicitation treatments.

\begin{figure}
\centering
\caption{Differences in the quality of the submitted project proposals}
\label{fig: quality}
\includegraphics{../figs/quality.pdf}
\end{figure}

To formally test this hypothesis, we aggregate mean ratings for each
proposal and regress these aggregate measures on solicitation treatment
dummies. The regression results (not reported) show only an
insignificant relationship between ratings and solicitation treatments.
The treatment coefficients are all insignificantly different from zero,
with the linear model not significantly different from a constant model
(an overall F-test gives a p-value of 0.609).

We also examine the distribution of ratings as generated by treatments
with no aggregation.\footnote{The analysis on the aggregated data
  crucially relies on the assumption that an increment in a proposal's
  quality as measured by an increase in ratings from \(v\) to \(v+1\) is
  the same for any value \(v\).} We have over 12,000 ratings, providing
a very sensitive test for differences across treatments. Using a
Pearson's Chi-squared test we find that the hypothesis of dependence
between the distribution of ratings and the treatments is \emph{not}
quite significant at the 10 percent level (p-value of 0.103). Driving
the p-value is a less than \(2\) percent difference between the
proportion of 5's in the WPLACE treatment versus the other
distributions, which is probably due to outliers (the winning proposal
was in the WPLACE treatment). Taken together with the fact that our
sample is large, we have strong evidence suggesting that there are no
(economically meaningful) differences in the quality of project
proposals across treatments and in particular no evidence of a quantity
versus quality trade-off up to the resolution of the five-point
scale.\footnote{One may worry that such binning is a fairly coarse
  measure of quality. In particular, effects concentrated in the upper
  tail of the distribution may not be detected. For example, comparing
  the ratings of proposals A, B, C and D with hypothetical true
  qualities of 3, 4, 5, and 10 stars respectively. Under a five-point
  scale rating system, proposals A and B can be distinguished, but C and
  D cannot be distinguished. Hence, one needs to be very cautious in
  interpreting these results as evidence against quality effects in
  general.}

\subsubsection{Quality assessed by
managers}\label{quality-assessed-by-managers}

One potential limit of assessing quality only on the basis of peer
ratings is that the employees might have a different view of a
proposal's quality than executives (due, for instance, to a misalignment
of incentives). Indeed, to ensure alignment between managerial goals and
the peer assessment, all project proposals were further vetted by the
HTL staff before being considered for implementation funding. So, we now
focus on the outcomes of this vetting process to investigate more
broadly the presence of treatment effects on the quality of project
proposals.

The vetting process conducted by the HTL staff resulted in 93 proposals
being scored on a scale from 1 to 100 points with the authors of the
best 29 proposals invited to submit implementation plans. The remaining
\texttt{113\ -\ \ nrow(quality2)} proposals were excluded (and received
a score of zero) either because flagged as inappropriate for funding or
because the proponent manifested no intention to participate in the
implementation phase (a \texttt{ft\$method} finds no association between
proposals excluded and treatments with a p-value of
\texttt{ft\$p.value}).

The Spearman's rank correlation coefficient between the scores given by
the HTL staff and the average peer ratings is relatively high
(cor=0.198), indicating good agreement between our two measures of
quality. As for the assessments made by peers, we find no treatment
effects on quality using the scores by the management: a linear
regression on the score of proposals against treatment dummies gives
coefficients that are all insignificantly different from zero, with the
linear model not significantly different from a constant model (F-test;
p=0.378).

We also find no treatment differences in the percentage of submitters
being selected and invited by HTL staff to present additional
implementation plans: (a \texttt{ft\$method} gives a p-value of
\texttt{ft\$p.value}). Although not significant, employees who made
project proposals in the FUND solicitation treatment are \emph{less}
likely to be selected as finalist than the others (only 1 out of 7 in
the FUND treatment were selected and invited by the HTL staff),
providing additional evidence of a no quantity versus quality trade-off,
as discussed before.

\subsubsection{Quality assessed by higher word
counts}\label{quality-assessed-by-higher-word-counts}

Following prior research indicating that higher total word counts
reflect higher quality \citep{blumenstock2008size}, we also look at
differences in the word counts of a submission. We find that most
submissions are below 200 words with little differences between the
treatments. Testing for a significant linear regression relationship
between the length of submissions and treatment dummies returned an
overall insignificant result (p=.43, F-test), which is consistent with
our previous assertion of little, or no, differences in quality across
treatments.

\subsection{Content of the project
proposals}\label{content-of-the-project-proposals}

While we have shown little differences in the overall project quality
across solicitation treatments, it is easy to think of ways in which a
solicitation may affect the \emph{content} of the submitted proposals,
while keeping the quality constant. For example, different solicitations
may lead proponents to think about different kinds of problems or,
indirectly, through the sorting of proponents with differing needs or
knowledge of the problems inside the organization.

Before examining specific differences in the content of proposals, it is
important to point out that the proposed projects broadly conformed to
the stated goals of the contest, which was to improve Heart Center
operations by identifying problem areas and potential solutions. For
example, one project proposal that received high peer ratings was to
create a platform for patients to electronically review and update their
medicine list in the office prior to seeing the physician. Another was
to develop a smartphone application showing a patient's itinerary for
the day providing a guide from one test or appointment to another. This
suggests the aligning with improving the work processes within the
organization or providing high-quality patient care. Nevertheless, other
contest organizers may have varying goals and be concerned about
different aspects of the submissions.

To examine additional dimensions of submission content, we now study the
\emph{area of focus} of the submissions. Of particular interest is
understanding whether different wordings used in the general
encouragement solicitation (either towards improving the workplace or
targeting the wellbeing of patients) induce employees to concentrate on
different categories of interventions. Members of the HTL categorized
each project proposal into one of seven areas of focus (Table
\ref{tab: area-of-focus}): three categories (``Care coordination'',
``Staff workflow'', ``Workplace'') identified improvements for the
workplace, other three (``Information and access'', ``Patient care'',
and ``Quality and Safety'') focused on improvements centered around
patients, and another one (``Surgical tools and support to research'')
categorized projects developing tools to support scientific research.

\begin{figure}
\caption{Differences in the content of the submitted project proposals}
\label{fig: content}
\includegraphics{../figs/areas.pdf}
\begin{tablenotes}
The four panels show the proportions of submitted project proposals in each area of focus (a=Surgical tools and support to research,b=Quality and safety ,c=Workplace,d=Staff workflow,e=Care Coordination,f=Information and access,g=Patient support) for each solicitation treatment. The proportions of the WPLACE treatment are used as a reference in all panels.
\end{tablenotes}
\end{figure}

The proportions of submitted project proposals in each area of focus
(Figure \ref{fig: content}) exhibit very similar patterns for the WPLACE
and PRIZE solicitation treatments and different and uncorrelated
patterns for the other treatments; suggesting the presence of treatment
effects. We test to determine the overall association between these
proportions and our solicitation treatments with a Fisher's Exact Test
for Count Data with simulated p-value (based on 50000 replicates).
Results show a significant (p=0.088) association at the 90 percent
level, providing thus evidence that the variation in the content of the
submitted proposals is, at least in part, attributable to our
experimental solicitations.

To test which areas of focus is affected by our treatment, we regress
the probability of a project proposal being in a given category against
solicitation treatment dummies. We use an F-test where the null
hypothesis tested is that all the treatment effects have a zero effect
on the probability of the proposal being in a given category. The
results show that project proposals in the PCARE solicitation treatment
are less likely to fall in the ``Quality and Safety'' category; and
project proposals in the FUND solicitation treatment are less likely to
fall in the ``Information and access'' category.

Although it is difficult to interpret these results because our model
does not provide any prediction on the content of proposals, they
indicate a possible trade-off between stimulating participation via
solicitation and inducing selection in the type of contributions to the
public good, which complicates the analysis of incentives for public
goods inside organizations beyond what the current literature
anticipates.

\subsection{Estimating social
preferences}\label{estimating-social-preferences}

The analysis above has shown that our solicitation treatments had both
positive and negative effect on participation, with no effects on
quality. But what can we say about the employees' preferences towards
the common goal of improving the organization?

To gauge the magnitude of underlying preferences for contributing to the
organization, we now use the experimental data to calibrate the
theoretical model discussed before (Section
\ref{conceptual-framework-and-predictions}). Following the
mixed-strategy equilibrium of the model, the theoretical probability of
contributing must be proportional to the expected value of winning,
\(R\), the underlying preferences towards the public good, \(\gamma\),
the marginal costs of contributing, \(c\), and the number of agents,
\(n\).

We assume the cost of making a submission \(c\) is the same in each
treatment, which seems a reasonable assumption given everyone is asked
to perform the exact same task (submitting a project proposal) and the
submission procedure is identical. Then we derive a structural
relationship between the observed difference in the probability of
contributing \(\Delta p\) and the difference in the expected rewards
from winning \(\Delta R\) between the treatments:\footnote{This equation
  can be obtained by following these steps. First, we approximate the
  profit equating condition \eqref{eq: mixed-strategy} to a linear
  function by noticing that the \(1/(1-(1-p)^n)\) approximates one for
  \(n\) large enough and \(p\) sufficiently small. Second, we solve for
  \(p\) and we simplify using the definitions of \(\Delta p\) and
  \(\Delta R\).}

\begin{equation}
  \label{eq: delta}
  \Delta p \approx\frac{\Delta R}{n (c - \gamma)}.
\end{equation}

(Throughout this section we will consider \(\delta=0\) ignoring the
distinction between impure and pure altruism.) By solving for the net
cost of contributing \(\Delta c=c-\gamma\), we get

\begin{equation}
  \label{eq: gamma}
  \Delta c\approx  \Delta R / (n\Delta p). 
\end{equation}

This implies that the net cost of a submission (the material cost of
submitting net of the individual preference for the public good) must be
proportional to the ratio between the difference in rewards and the
difference in the probability of submitting. Although we do not observe
the levels of \(R\) in each treatment, we approximate the difference of
rewards between the PRIZE and the other conditions by the pecuniary
value of the reward, which has its upper bound in the highest price that
can be paid for an iPad mini (\$350).\footnote{The price paid by the
  Heart Center was \$239 at the end of 2014 (including shipping cost).
  Other popular models (those with cellular data and large storage)
  could cost as high as \$350. Agents, however, were not aware of the
  specific model used for the competition and of the price paid. So, the
  value of \$350 is very conservative.} And we round the competitors up
to \(n=1000\) to be conservative. Finally, by substituting those
calibrated values into equation \eqref{eq: gamma} along with the
empirical difference in participation rates between the PRIZE and the
other treatments (\(\Delta p=0.037\)), we get an estimate of the
magnitude of net cost which is \(\Delta c=\) \$350/(1000*0.037)=\$9.5.

We can now compare the estimated net cost of contributing against the
hourly wage of the median staff member, which is \$40 (the median income
per hour of a nurse according to the Bureau of Labor Statistics). This
comparison shows that the wage of the median staff member per hour is
more than 4 times \emph{higher} than the net cost of making a
contribution. Since the time to write and submit a project proposal will
likely take one hour of work or more, the net cost of contributing
appears too low to be consistent with no preferences towards the public
good. In other words, by comparing the calibrated net cost of
contributing against the monetary cost of an hour of work, we find a
negative gap that appears too large to be explained without assuming
some non-monetary motivations acting as a compensating force.

\section{Summary and conclusions}\label{summary-and-conclusions}

We conduct a natural field experiment to provide evidence on the
effectiveness of personal rewards and mission-oriented incentives within
an internal innovation contest at a medical organization seeking
contribution of public goods (i.e., projects for organizational
improvement) from its more than 1200 employees. The experiment tested
incentives for contributing by manipulating the content of personalized
emails soliciting staff participation, such as (\(i\)) offering awards
for winning submissions, (\(ii\)) funding for implementation of own
projects, and appealing to intrinsic motivations towards (\(iii\))
improving patient care or (\(iv\)) the workplace.

We find that solicited employees who are offered personal rewards for
winning submissions have a higher propensity to participate in the
contest than those who are offered funding to implement own projects or
are encouraged appealing to their intrinsic motivations towards the
patients and coworkers. The observed participation differences, we find,
are without changing the quality of the submissions as judged by peer
ratings and the evaluations conducted by the management, for which there
is good agreement (high positive correlation) suggesting incentives
being sufficiently aligned. This means that the higher propensity to
participate of solicited employees who are offered prizes does not seem
to be driven by low-quality submissions. In addition, we provide
evidence that the effect of a prize competition on participation goes
beyond the actual value of the prize itself, thus suggesting employees
have internalized some of the public good effects of contributing.

Another key finding is that the opportunity of funding is a poor
incentive. Solicited employees who are offered funding to lead the
implementation of one's own submitted project proposal have a lower
propensity to participate than those who are offered prizes for winning
submissions and those who are not offered any particular reward (besides
the value of improving the organization for its own good). A potential
trade-off may occur with employee participation suffering at the
expenses of high-quality proposals. Contrary to this possibility, we
find that the lower employee participation is without changing the
quality of the submissions as measured by the assessment of peers and
the management, thus it does not seem to be driven by high-quality
submissions.

While employees who are asked to contribute for intrinsic motives, such
as those towards providing better care to their patients and improving
the workplace environment, have equally strong participation propensity
on average, we find that responses appear sensitive to the gender of the
solicited person. Women's participation is greater when emphasizing the
patient care whereas men's participation is significantly lower,
controlling for the profession and position inside the organization.
This finding suggests that gender may be an important factor influencing
sensitivity of responses to solicitations concerning the organizational
mission. At the same time, only an insignificant gender-based
differences with respect to participation associated with the offering
of prizes was found: women's participation was slightly higher but not
significant than men's, all else being equal. This evidence indicates
that gender differences in preferences, such as competitive inclinations
or risk aversion, may not exert great influence on responses of workers
to contests inside organizations.

We believe these results have three main implications for comparable
organizations and, more broadly, the internal provision of public goods.

The first implication is that announcing a competition for an individual
prize foster workers' participation in organizational tasks beyond the
value of the awarded prize itself. That is, prizes generate two opposing
externalities that help workers internalize the public good effects of
their organizational contributions. This result is important because it
highlights a relatively less understood function of contests that is to
mitigate the free riding incentives on organizational tasks.

A second implication is that offering the opportunity to lead collective
projects can exacerbate the free riding incentives. This result may
appear contrary to intuition. In theory, one may benefit more from
leading a project than winning an iPad. For instance, one may use the
opportunity to signal project management skills to the management aiming
for a career advancement; or steer some of the resources towards assets
or problems that are relatively more beneficial to his or her situation
compared to the rest of the organization. If so, why a negative result?
We believe that the private benefits from winning were negligible in our
setting. First, the opportunity for a career advancement is small
because medical staff gets promoted on the basis of other parameters
(e.g., the quality of care provided). Second, the peer evaluation and
vetting by the management ensure that the winning contributions yield as
distributed benefits as possible. These aspects may have eliminated the
possible private benefits from leading a project, resulting in poor
participation rates. This result is important because projects need to
be lead by someone and making more resources available does not seem to
increase volunteers.

A third implication is that participation in organizational tasks is
sometimes triggered by mission-based preferences. Although we find
evidence that these preferences can be an effective incentive, we also
find gender-based selection effects that are difficult to predict
ex-ante. Our experiment does not provide any insights to better
interpret these differences. But a large literature has investigated
gender-based difference in preferences \citep[see][ for a
review]{croson2009gender} or difference in self-stereotypes
\citep{coffman2014evidence} that could explain some of these effects.
Yet, more experimentation is needed to understand the different drivers
in the field inside organizations.

A few limitations of this study deserve consideration. The first is that
the validity of our causal interpretation of the results rests on a few
conventional assumptions \citep{rubin1974estimating}. These include the
``no interference between units'' assumption. In our study, it is
possible that communication among staff assigned to different treatment
arms could have influenced decisions to participate. The magnitude of
this interference would depend on intensity of staff communication and
the density of social interactions. Both of which should be small
because (i) an individual competition may provide only weak incentives
for information sharing and (ii) the staff members are scattered across
multiple buildings on the hospital campus. Even so, a potential
inference bias may alter the results towards a null effect as
differences in employee participation should converge towards zero when
communication spreads the content of the different email solicitations.
This goes against our results. Moreover, by looking at the temporal
dynamics of submissions, we find no indication of a convergence in the
participation rates. Hence, the assumption of no interference seems
appropriate.

Another potential limitation is that staff members may have left the
solicitation email that was sent to them unopened or unread, thus
non-complying with the assigned solicitation treatment. As this kind of
noncompliance is almost entirely unobserved,\footnote{The email was sent
  using the internal messaging system of the Heart Center, which, at the
  time, was not collecting individual analytics.} the analysis follows
an \emph{Intention-To-Treat} (ITT) approach, discarding entirely any
information about the solicitation treatment actually received. The main
drawback of an ITT analysis is that it does not answer questions about
causal effects of the content of the solicitation itself, only about
causal effects of the assignment to a solicitation treatment.

Finally, our results have implications that extend beyond the specific
organization under study. While the choice of focusing on health care
workers may limit the generalizability of our results in some respect,
it should be noted that in the US alone health care spending accounts
for 17 percent of the GDP (in 2015). And, more generally, our study
results are also directly applicable to a variety of other professions
exposed to a public good dilemma in a mission-driven environment (e.g.,
teachers, public servants, researchers). In all these settings, our
study suggests that contests soliciting employee contributions and
awarding an individual prize to the winning contribution appear an
effective way to foster the internal provision of public goods inside
organizations.

\renewcommand\refname{References}
\bibliography{refs.bib}

\end{document}